\usepackage{amsmath,amssymb,amsthm,amsfonts}
\usepackage[utf8]{inputenc}
\usepackage{algorithmic,algorithm}
\usepackage{graphics,graphicx}
\usepackage{pgfplots,tikz}
\usepackage{enumerate}
\usepackage[ngerman]{babel}
\usepackage{eurosym}
\usepackage{hyperref}
\usepackage{caption, subcaption}

\floatname{algorithm}{Algorithmus}
\renewcommand{\algorithmicrequire}{\textbf{Input:}}
\renewcommand{\algorithmicensure}{\textbf{Output:}}

\textheight23cm
\topmargin-15mm
\oddsidemargin-5mm    %  -10mm
\textwidth17cm    %   18.8cm
\footskip0pt
\thispagestyle{empty}
\parindent0mm
\parskip0ex
\parskip0ex

\makeatletter
\DeclareOldFontCommand{\rm}{\normalfont\rmfamily}{\mathrm}
\DeclareOldFontCommand{\sf}{\normalfont\sffamily}{\mathsf}
\DeclareOldFontCommand{\tt}{\normalfont\ttfamily}{\mathtt}
\DeclareOldFontCommand{\bf}{\normalfont\bfseries}{\mathbf}
\DeclareOldFontCommand{\it}{\normalfont\itshape}{\mathit}
\DeclareOldFontCommand{\sl}{\normalfont\slshape}{\@nomath\sl}
\DeclareOldFontCommand{\sc}{\normalfont\scshape}{\@nomath\sc}
\makeatother

\newcommand{\Aufgabe}[2][]{\par\bigskip{\sf\bfseries Aufgabe #2#1:}}

\begin{document}
{\sf Universität Hamburg \hfill Studi 1, Matrikelnummer 1 \\ Wintersemester 2020/21 \hfill Studi 2, Matrikelnummer 2 \\ \phantom{x} \hfill Studi 3, Matrikelnummer 3}
\begin{center}
\ifthenelse{\equal{\nr}{no}}{\Large\sf\bfseries \kopf}{\Large\sf\bfseries Optimierung f\"ur Studierende der Informatik -- \nr.~\kopf}
\end{center}

\renewcommand{\tilde}{\widetilde}
\renewcommand{\hat}{\widehat}
\newcommand{\ri}{\mathrm{i}}
\renewcommand{\H}{\mathsf{H}}
\newcommand{\T}{\mathsf{T}}
