\def\nr{2. Aufgabenblatt}
\def\kopf{\\\hfill\normalsize\mdseries}
\documentclass[11pt,a4paper,fleqn]{scrartcl}
\usepackage{eurosym}
%\usepackage{a4kopka}
\usepackage{amsmath,amssymb,amsthm,amsfonts}
\usepackage[utf8]{inputenc}
\usepackage{algorithmic,algorithm}
\usepackage{graphics,graphicx}
\usepackage{pgfplots,tikz}
\usepackage{enumerate}
\usepackage[ngerman]{babel}
% \usepackage[software]{mymacros}
%\usepackage{matrix}
\usepackage{hyperref}
% \usepackage{caption}
\usepackage{caption, subcaption}

\floatname{algorithm}{Algorithmus}
\renewcommand{\algorithmicrequire}{\textbf{Input:}}
\renewcommand{\algorithmicensure}{\textbf{Output:}}

%\usepackage{enumitem} 
%\textheight25cm
\textheight23cm
\topmargin-15mm
\oddsidemargin-5mm    %  -10mm
\textwidth17cm    %   18.8cm
\footskip0pt
\thispagestyle{empty}
\parindent0mm
\parskip0ex
\parskip0ex

\makeatletter
\DeclareOldFontCommand{\rm}{\normalfont\rmfamily}{\mathrm}
\DeclareOldFontCommand{\sf}{\normalfont\sffamily}{\mathsf}
\DeclareOldFontCommand{\tt}{\normalfont\ttfamily}{\mathtt}
\DeclareOldFontCommand{\bf}{\normalfont\bfseries}{\mathbf}
\DeclareOldFontCommand{\it}{\normalfont\itshape}{\mathit}
\DeclareOldFontCommand{\sl}{\normalfont\slshape}{\@nomath\sl}
\DeclareOldFontCommand{\sc}{\normalfont\scshape}{\@nomath\sc}
\makeatother

% \newcommand{\cg}[1]{{\color{blue} #1}}
% \newcommand{\cb}[1]{{\color{green} #1}}
% \newcommand{\cred}[1]{{\color{red} #1}}
% \newcommand{\cc}[1]{{\color{cyan} #1}}
% \newcommand{\cm}[1]{{\color{magenta} #1}}

\newcommand{\Aufgabe}[2][]{\par\bigskip{\sf\bfseries Aufgabe #2#1:}}
%\hspace{3em}{\small(#2 point\ifthenelse{#2>1}{s}{})}}\par\smallskip}
%\newcommand\aufgabe[2][~]{\par\bigskip{\sf\bfseries Aufgabe #1
%    \hspace{3em} \ifthenelse{\equal{#2}{~}}{}{(#2)}}\par\smallskip}
\usepackage{mymacros}

\begin{document}
{\sf Universit\"at Hamburg \hfill Wintersemester 2020/21 \\ Fachbereich Mathematik \\ Dr. Matthias Voigt}
\begin{center}
\ifthenelse{\equal{\nr}{no}}{\Large\sf\bfseries \kopf}{\Large\sf\bfseries Optimierung f\"ur Studierende der Informatik -- \nr.~\kopf}
\end{center}

\renewcommand{\tilde}{\widetilde}
\renewcommand{\hat}{\widehat}
\newcommand{\ri}{\mathrm{i}}
\renewcommand{\H}{\mathsf{H}}
\newcommand{\T}{\mathsf{T}}


\subsection*{Präsenzaufgaben am 16./17.11.2020}

\Aufgabe[ (Simplexverfahren)]{P1}
Schauen Sie sich die \textbf{Übersicht zum Simplexverfahren} an und beantworten Sie die folgenden Fragen:
\begin{enumerate}[i)]
% Aufgabe P-1 (i)
\item Wie kommt das Starttableau zustande?
% Aufgabe P-1 (ii)
\item Weshalb wurde in der 1. Iteration $x_1$ als Eingangsvariable gewählt? Wie kommt die Wahl von $x_4$ als Ausgangsvariable zustande?
% Aufgabe P-1 (iii)
\item Als Ergebnis der 1. Iteration erhält man ein neues Tableau. Wie kommt die 1. Zeile in diesem Tableau zustande? Wie ergeben sich die übrigen Zeilen (einschließlich der $z$-Zeile)?
% Aufgabe P-1 (iv)
\item Woran erkennt man, dass das Tableau am Ende der 2. Iteration optimal ist? Wie ergibt sich am Schluss die optimale Lösung?
% Aufgabe P-1 (v)
\item Können Sie anhand der $z$-Zeile im letzten Tableau begründen, weshalb die gefundene Lösung tatsächlich optimal ist?
\end{enumerate}

\Aufgabe[ (Simplexverfahren am Beispiel)]{P2}
Wir betrachten das folgende LP-Problem, das mit dem Simplexverfahren gelöst werden soll; dabei ist genau wie in der \textbf{Übersicht zum Simplexverfahren} vorzugehen. Insbesondere ist am Ende jeder Iteration das neue Tableau noch einmal übersichtlich hinzuschreiben.
\begin{align*}
\begin{alignedat}{5}
& \text{maximiere } & 4x_1 &\ + &\ x_2 &\ - &\ 3x_3 & & \\
& \rlap{unter den Nebenbedingungen} & & & & & & & \\
&& -2x_1 &\ - &\ 2x_2 &\ + &\ 3x_3 &\ \leq &\ 2, \\
&&  2x_1 &\ + &\  x_2 &\   &\      &\ \leq &\ 5, \\
&&   x_1 &\ - &\  x_2 &\ - &\ 5x_3 &\ \leq &\ 4, \\
&&&&&& \llap{$x_1,x_2,x_3$} &\ \geq &\ 0.
\end{alignedat}
\end{align*}

\subsection*{Hausaufgaben bis zum 25.11.2020 (12:00 Uhr)}
\emph{Bitte reichen Sie Ihre Hausaufgaben in festen Zweier- oder Dreiergruppen bei Moodle ein. Bitte laden Sie ausschließlich \textbf{PDF-Dokumente} hoch, andernfalls können Ihre Hausaufgaben nicht korrigiert werden.}

\Aufgabe[ (Simplexverfahren, 6 Punkte)]{H1}
Lösen Sie die folgenden LP-Probleme mit dem Simplexverfahren:
\begin{enumerate}[a)]
% Aufgabe H-1a
\item \begin{align*}
\begin{alignedat}{6}
& \text{maximiere } & -2x_1 &\ + &\ 3x_2 &\ + &\ \frac{1}{2}x_3 & & \\
& \rlap{unter den Nebenbedingungen} & & & & & & & \\
&& -2x_1 &\ + &\ 3x_2 &\ - &\  x_3 &\ \leq &\ 4, \\
&&   x_1 &    &       &\ + &\ 2x_3 &\ \leq &\ 10, \\
&&  -x_1 &\ + &\  x_2 &    &       &\ \leq &\ 2, \\
&& & & & & \llap{$x_1, x_2, x_3$} &\ \geq &\ 0. 
\end{alignedat}
\end{align*}

% Aufgabe H-1b
\item \begin{align*}
\begin{alignedat}{6}
& \text{maximiere } & 3x_1 &\ + &\ x_2 &\ - &\ 10x_3 &\ - &\ 9x_4 & & \\
& \rlap{unter den Nebenbedingungen} & & & & & & & & & \\
&& x_1 &\ - &\ x_2 &\ - &\ x_3 &\ + &\ x_4 &\ \leq &\ 2,\ \\
&& x_1 &\ + &\ x_2 &\ + &\ 3x_3 &\ + &\  x_4 &\ \leq &\ 1,\ \\
&& & & & & & & \llap{$x_1,x_2,x_3,x_4$} &\ \geq &\ 0.\
\end{alignedat}
\end{align*}
\end{enumerate}

\Aufgabe[ (Simplexverfahren und unbeschränkte Probleme, 4 Punkte)]{H2}
\begin{enumerate}[a)]
	% Aufgabe H-2a
	\item Lösen Sie die folgende Aufgabe mit dem Simplexverfahren:
	\begin{align*}
	\begin{alignedat}{5}
	& \text{maximiere } & -x_1 &\ + &\ 3x_2 &\ + &\ x_3 & & \\
	& \rlap{unter den Nebenbedingungen} & & & & & & & \\
	&& -x_1 &\ + &\ 2x_2 &\ + &\ 2x_3 &\ \leq &\ 4,\ \\
	&&  x_1 &\ - &\ 2x_2 &\ + &\ 3x_3 &\ \leq &\ 4,\ \\
	&&  x_1 &\ - &\ 3x_2 &\ + &\  x_3 &\ \leq &\ 4,\ \\
	&&&&&& \llap{$x_1,x_2,x_3$} &\ \geq &\ 0.\
	\end{alignedat}
	\end{align*}
	
	% Aufgabe P-1b
	\item Falls das Verfahren mit dem Ergebnis \textbf{unbeschränkt} terminiert, so gebe man zulässige Lösungen an, für die der Zielfunktionswert $z$ die folgenden Werte annimmt: $z=10$, $z=100$ sowie $z=1000$.
\end{enumerate}
\end{document}
