\def\nr{1. Aufgabenblatt}
\def\kopf{\\\hfill\normalsize\mdseries}
\documentclass[11pt,a4paper,fleqn]{scrartcl}
\usepackage{eurosym}
%\usepackage{a4kopka}
\usepackage{amsmath,amssymb,amsthm,amsfonts}
\usepackage[utf8]{inputenc}
\usepackage{algorithmic,algorithm}
\usepackage{graphics,graphicx}
\usepackage{pgfplots,tikz}
\usepackage{enumerate}
\usepackage[ngerman]{babel}
% \usepackage[software]{mymacros}
%\usepackage{matrix}
\usepackage{hyperref}
% \usepackage{caption}
\usepackage{caption, subcaption}

\floatname{algorithm}{Algorithmus}
\renewcommand{\algorithmicrequire}{\textbf{Input:}}
\renewcommand{\algorithmicensure}{\textbf{Output:}}

%\usepackage{enumitem} 
%\textheight25cm
\textheight23cm
\topmargin-15mm
\oddsidemargin-5mm    %  -10mm
\textwidth17cm    %   18.8cm
\footskip0pt
\thispagestyle{empty}
\parindent0mm
\parskip0ex
\parskip0ex

\makeatletter
\DeclareOldFontCommand{\rm}{\normalfont\rmfamily}{\mathrm}
\DeclareOldFontCommand{\sf}{\normalfont\sffamily}{\mathsf}
\DeclareOldFontCommand{\tt}{\normalfont\ttfamily}{\mathtt}
\DeclareOldFontCommand{\bf}{\normalfont\bfseries}{\mathbf}
\DeclareOldFontCommand{\it}{\normalfont\itshape}{\mathit}
\DeclareOldFontCommand{\sl}{\normalfont\slshape}{\@nomath\sl}
\DeclareOldFontCommand{\sc}{\normalfont\scshape}{\@nomath\sc}
\makeatother

% \newcommand{\cg}[1]{{\color{blue} #1}}
% \newcommand{\cb}[1]{{\color{green} #1}}
% \newcommand{\cred}[1]{{\color{red} #1}}
% \newcommand{\cc}[1]{{\color{cyan} #1}}
% \newcommand{\cm}[1]{{\color{magenta} #1}}

\newcommand{\Aufgabe}[2][]{\par\bigskip{\sf\bfseries Aufgabe #2#1:}}
%\hspace{3em}{\small(#2 point\ifthenelse{#2>1}{s}{})}}\par\smallskip}
%\newcommand\aufgabe[2][~]{\par\bigskip{\sf\bfseries Aufgabe #1
%    \hspace{3em} \ifthenelse{\equal{#2}{~}}{}{(#2)}}\par\smallskip}
\usepackage{mymacros}

\begin{document}
{\sf Universit\"at Hamburg \hfill Wintersemester 2020/21 \\ Fachbereich Mathematik \\ Dr. Matthias Voigt}
\begin{center}
\ifthenelse{\equal{\nr}{no}}{\Large\sf\bfseries \kopf}{\Large\sf\bfseries Optimierung f\"ur Studierende der Informatik -- \nr.~\kopf}
\end{center}

\renewcommand{\tilde}{\widetilde}
\renewcommand{\hat}{\widehat}
\newcommand{\ri}{\mathrm{i}}
\renewcommand{\H}{\mathsf{H}}
\newcommand{\T}{\mathsf{T}}


\subsection*{Präsenzaufgaben am 09./10.11.2020}

\Aufgabe[ (LP-Probleme in Standardform)]{P1}
Bei welchen der folgenden LP-Probleme handelt es sich nicht um ein Problem in Standardform?
\begin{enumerate}[i)]
% Aufgabe P-1 (i)
\item \begin{align*}
\begin{alignedat}{5}
& \text{maximiere } & 3x_1 &\ + &\ 4x_2 &\ - &\ 5x_3 & & \\
& \rlap{unter den Nebenbedingungen} & & & & & & & \\
&& 4x_1 &\ + &\ 3x_2 &\ + &\ 5x_3 &\ \geq &\  8, \\
&& 6x_1 &\ + &\  x_2 &\ - &\ 6x_3 &\ =    &\  5, \\
&& x_1 &\ + &\ 8x_2 &\ + &\ 8x_3 &\ \leq &\ 21, \\
&& & & & & \llap{$x_1, x_2, x_3$} &\ \geq &\ 0. 
\end{alignedat}
\end{align*}

% Aufgabe P-1 (ii)
\item \begin{align*}
\begin{alignedat}{6}
& \text{minimiere } & 3x_1 &\ + &\ x_2 &\ + &\ 4x_3 &\ + &\ x_4 & & \\
& \rlap{unter den Nebenbedingungen} & & & & & & & & & \\
&& 9x_1 &\ + &\ 2x_2 &\ + &\ 6x_3 &\ + &\ 5x_4 &\ \leq &\  7, \\
&& 8x_1 &\ + &\ 9x_2 &\ + &\ 7x_3 &\ + &\ 3x_4 &\ \leq &\  2, \\
&& & & & & & & \llap{$x_1, x_2, x_3$} &\ \geq &\ 0.
\end{alignedat}
\end{align*}

% Aufgabe P-1 (iii)
\item \begin{align*}
\begin{alignedat}{5}
& \text{maximiere } & 8x_1 &\ - &\ 3x_2 &\ - &\ 4x_3 & & \\
& \rlap{unter den Nebenbedingungen} & & & & & & & \\
&& 3x_1 &\ + &\  x_2 &\ + &\  x_3 &\ \leq &\  5, \\
&& 9x_1 &\ + &\ 5x_2 &    &       &\ \leq &\ -2, \\
&& & & & & \llap{$x_1, x_2, x_3$} &\ \geq &\ 0.
\end{alignedat}
\end{align*}
\end{enumerate}

\Aufgabe[ (Überführung in Standardform)]{P2}
Überführen Sie die Probleme aus Aufgabe P1, die nicht in Standardform vorliegen, in eine solche.

\Aufgabe[ (Unlösbare und unbeschränkte Probleme)]{P3}
Handelt es sich bei den folgenden LP-Problemen um unlösbare Probleme? Ist eines der Probleme unbeschränkt?
\begin{enumerate}[i)]
% Aufgabe P-3 (i)
\item \begin{align*}
\begin{alignedat}{6}
& \text{maximiere } & 3x_1 &\ - &\ x_2 &\ - &\ x_3 &\ - &\ x_4 & & \\
& \rlap{unter den Nebenbedingungen} & & & & & & & & & \\
&&   x_1 &\ + &\  x_2 &\ + &\  x_3 &\ + &\  x_4 &\ \leq &\   2, \\
&& -4x_1 &\ - &\ 4x_2 &\ - &\ 4x_3 &\ - &\ 4x_4 &\ \leq &\ -10, \\
&& & & & & & & \llap{$x_1, x_2, x_3, x_4$} &\ \geq &\ 0.
\end{alignedat}
\end{align*}

% Aufgabe P-3 (ii)
\item \begin{align*}
\begin{alignedat}{6}
& \text{maximiere } & -x_1 &\ - &\ x_2 &\ - &\ x_3 &\ + &\ x_4 & & \\
& \rlap{unter den Nebenbedingungen} & & & & & & & & & \\
&&   x_1 &\ + &\  x_2 &\ + &\  x_3 &\ - &\ 2x_4 &\ \leq &\ -1, \\
&& -2x_1 &\ - &\ 2x_2 &\ + &\ 5x_3 &\ - &\  x_4 &\ \leq &\ -3, \\
&& & & & & & & \llap{$x_1, x_2, x_3, x_4$} &\ \geq &\ 0.
\end{alignedat}
\end{align*}
\end{enumerate}

\Aufgabe[ (Graphische Methode)]{P4}
Lösen Sie das folgende LP-Problem mithilfe der graphischen Methode: 
\begin{align*}
\begin{alignedat}{4}
& \text{maximiere } & x_1 &\ + &\ x_2 & & \\
& \rlap{unter den Nebenbedingungen} & & & & & \\
&& 2x_1 &\ - &\ 5x_2 &\ \leq &\  2,\ \\
&& -x_1 &\ + &\ 4x_2 &\ \leq &\  8,\ \\
&&  x_1 &\ + &\ 2x_2 &\ \leq &\ 10,\ \\
&& & & \llap{$x_1,x_2$} &\ \geq &\ 0.\
\end{alignedat}
\end{align*}

\subsection*{Hausaufgaben bis zum 18.11.2020 (12:00 Uhr)}
\emph{Bitte reichen Sie Ihre Hausaufgaben in festen Zweier- oder Dreiergruppen bei Moodle ein. Bitte laden Sie ausschließlich \textbf{PDF-Dokumente} hoch, andernfalls können Ihre Hausaufgaben nicht korrigiert werden.}

\Aufgabe[ (Überführung in Standardform und graphische Methode, 4 Punkte)]{H1}
\begin{enumerate}[a)]
\item Überführen Sie das folgenden Problem in Standardform: \begin{align*}
\begin{alignedat}{6}
& \text{maximiere } & -6x_1 &\ - &\ 1x_2 &\ + &\ x_3 &\ + &\ 2x_4 & & \\
& \rlap{unter den Nebenbedingungen} & & & & & & & & & \\
&&  4x_1 &\ - &\ x_2 &\ + &\ 2x_3 &    &       &\ \geq &\  3, \\
&& -7x_1 &\ + &\ 2x_2 &\ - &\ x_3 &\ + &\  x_4 &\ \leq &\  3, \\
&&       &    &\ x_2 &\ + &\ 3x_3 &\ - &\ 4x_4 &\ =    &\  1, \\
&& 	     &    &      &    &       &    &\  x_4 &\ \leq &\  1, \\
&& & & & & & & \llap{$x_2,x_3,x_4$} &\ \geq &\ 0.
\end{alignedat}
\end{align*}
% Aufgabe H-1b
\item Lösen Sie das folgende Problem mit der graphischen Methode:
\begin{align*}
\begin{alignedat}{4}
& \text{maximiere } & x_1 &\ + &\ 3x_2 & & \\
& \rlap{unter den Nebenbedingungen} & & & & & \\
&&   x_1 &\ + &\  x_2 &\ \leq &\  7,\ \\
&&  2x_1 &\ + &\ 5x_2 &\ \leq &\  20,\ \\
&&  2x_1 &\ + &\ x_2 &\ \leq &\  12,\ \\
&& & & \llap{$x_1,x_2$} &\ \geq &\ 0.\
\end{alignedat}
\end{align*}
\end{enumerate}

\Aufgabe[ (Kantinenoptimierung, 3 Punkte)]{H2}
Ein Kantinenleiter hat folgendes Problem: Ein Erwachsener soll täglich mindestens 100g Fett, 120g Protein und 400g Kohlenhydrate aufnehmen. Nehmen Sie an, dass diese Forderungen erfüllt werden sollen und die folgenden Informationen beachtet werden sollen. Welche Waren sollten gekauft werden und wie viel sollte von jedem Gut gekauft werden, wenn die billigste Möglichkeit realisiert werden soll? Die Anzahl Gramm an Proteinen, Fett und Kohlenhydraten in 100g einer Reihe von Nahrungsmitteln ist in folgender Tabelle gegeben:

\begin{center}
\begin{tabular}{l|c|c|c}
 & Protein & Fett & Kohlenhydrate \\ \hline
Hähnchen     & 29 &  8 &  0 \\
Fisch        & 41 &  9 &  0 \\
Backpflaumen &  5 &  4 & 37 \\
Weißbrot     &  7 &  4 & 58 \\
Käse         & 25 & 46 &  0 \\
Schwarzbrot  & 11 & 12 & 63 \\
Nüsse        &  9 & 50 &  4 \\
Margarine    &  0 & 89 &  0
\end{tabular}
\end{center}

Es soll eine Höchstgrenze von 80g Brot pro Tag und pro Person berücksichtigt werden und jede Person soll mindestens doppelt soviel Schwarz- wie Weißbrot verzehren.

Die Preise in {\euro} pro 100g werden für die verschiedenen Lebensmittel wie folgt angenommen:

\begin{center}
\begin{tabular}{c|c|c|c|c|c|c|c}
H & F & B & W & K & S & N & M \\ \hline
1.00 & 1.00 & 1.11 & 0.49 & 1.09 & 0.91 & 0.98 & 0.60
\end{tabular}
\end{center}

Formulieren Sie die Aufgabe, vor der der Kantinenleiter steht, als LP-Problem.
\Aufgabe[ (Eiscremeherstellung, 3 Punkte)]{H3}
Ein Eiscremehersteller produziert pro Tag 480 Einheiten von Eissorte $A$, 360 Einheiten von Sorte $B$ und 200 Einheiten von Sorte $C$. Jede dieser Eissorten kann in einem arbeitsaufwändigen Prozess verfeinert werden, wodurch Luxusvarianten entstehen. Pro Tag können in der regulären Arbeitszeit bis zu 420 Einheiten veredelt werden; darüber hinaus ist es möglich, mithilfe von Überstunden weitere 250 Einheiten zu veredeln -- allerdings zu erhöhten Kosten. Die Gewinne pro Einheit sind wie folgt:
	\[
	\begin{array}{c|c|c|c}
	& \text{einfach} 
	& \begin{array}{c} \text{in regulärer} \\ \text{Arbeitszeit veredelt} \end{array} 
	& \begin{array}{c} \text{im Rahmen von} \\ \text{Überstunden veredelt} \end{array} \\[1mm] \hline
	\text{Sorte } A & 7$ \euro$ & 15$ \euro$ & 10$ \euro$ \\
	\text{Sorte } B & 3$ \euro$ & 12$ \euro$ &  6$ \euro$ \\
	\text{Sorte } C & 1$ \euro$ & 11$ \euro$ & 10$ \euro$ 
	\end{array}
	\]
	% 7*185+15*240+10*55+265*3+60*12+35*6+45*1+11*60+95*10
	Beispielsweise führt der folgende Produktionsplan zu einem Gewinn von 8825 \euro:
	\[
	\begin{array}{c|c|c|c}
	& \text{einfach} 
	& \begin{array}{c} \text{in regulärer} \\ \text{Arbeitszeit veredelt} \end{array} 
	& \begin{array}{c} \text{im Rahmen von} \\ \text{Überstunden veredelt} \end{array} \\[1mm] \hline
	\text{Sorte } A & 185 & 240 &  55 \\[1mm]
	\text{Sorte } B & 265 &  60 &  35 \\[1mm]
	\text{Sorte } C &  45 &  60 &  95
	\end{array}
	\]
	
	Das Ziel ist, einen Produktionsplan zu finden, der den Gewinn maximiert. Formulieren Sie dieses Problem als LP-Problem in Standardform, wobei Sie mit 6 Variablen $x_1,\ldots,x_6$ auskommen sollen.

\end{document}
