\def\nr{5. Aufgabenblatt}
\def\kopf{\\\hfill\normalsize\mdseries}
\documentclass[11pt,a4paper,fleqn]{scrartcl}
\usepackage{eurosym}
%\usepackage{a4kopka}
\usepackage{amsmath,amssymb,amsthm,amsfonts}
\usepackage[utf8]{inputenc}
\usepackage{algorithmic,algorithm}
\usepackage{graphics,graphicx}
\usepackage{pgfplots,tikz}
\usepackage{enumerate}
\usepackage[ngerman]{babel}
% \usepackage[software]{mymacros}
%\usepackage{matrix}
\usepackage{hyperref}
% \usepackage{caption}
\usepackage{caption, subcaption}

\floatname{algorithm}{Algorithmus}
\renewcommand{\algorithmicrequire}{\textbf{Input:}}
\renewcommand{\algorithmicensure}{\textbf{Output:}}

%\usepackage{enumitem} 
%\textheight25cm
\textheight23cm
\topmargin-15mm
\oddsidemargin-5mm    %  -10mm
\textwidth17cm    %   18.8cm
\footskip0pt
\thispagestyle{empty}
\parindent0mm
\parskip0ex
\parskip0ex

\makeatletter
\DeclareOldFontCommand{\rm}{\normalfont\rmfamily}{\mathrm}
\DeclareOldFontCommand{\sf}{\normalfont\sffamily}{\mathsf}
\DeclareOldFontCommand{\tt}{\normalfont\ttfamily}{\mathtt}
\DeclareOldFontCommand{\bf}{\normalfont\bfseries}{\mathbf}
\DeclareOldFontCommand{\it}{\normalfont\itshape}{\mathit}
\DeclareOldFontCommand{\sl}{\normalfont\slshape}{\@nomath\sl}
\DeclareOldFontCommand{\sc}{\normalfont\scshape}{\@nomath\sc}
\makeatother

% \newcommand{\cg}[1]{{\color{blue} #1}}
% \newcommand{\cb}[1]{{\color{green} #1}}
% \newcommand{\cred}[1]{{\color{red} #1}}
% \newcommand{\cc}[1]{{\color{cyan} #1}}
% \newcommand{\cm}[1]{{\color{magenta} #1}}

\newcommand{\Aufgabe}[2][]{\par\bigskip{\sf\bfseries Aufgabe #2#1:}}
%\hspace{3em}{\small(#2 point\ifthenelse{#2>1}{s}{})}}\par\smallskip}
%\newcommand\aufgabe[2][~]{\par\bigskip{\sf\bfseries Aufgabe #1
%    \hspace{3em} \ifthenelse{\equal{#2}{~}}{}{(#2)}}\par\smallskip}
\usepackage{mymacros}

\begin{document}
{\sf Universit\"at Hamburg \hfill Wintersemester 2020/21 \\ Fachbereich Mathematik \\ Dr. Matthias Voigt}
\begin{center}
\ifthenelse{\equal{\nr}{no}}{\Large\sf\bfseries \kopf}{\Large\sf\bfseries Optimierung f\"ur Studierende der Informatik -- \nr.~\kopf}
\end{center}

\renewcommand{\tilde}{\widetilde}
\renewcommand{\hat}{\widehat}
\newcommand{\ri}{\mathrm{i}}
\renewcommand{\H}{\mathsf{H}}
\newcommand{\T}{\mathsf{T}}


\subsection*{Präsenzaufgaben am 07./08.12.2020}

\Aufgabe[ (duales Problem)]{P1}
Betrachten Sie das folgende LP-Problem in Standardform:
\begin{align}
\tag{P}
\begin{alignedat}{5}
& \text{maximiere } & 5x_1 &\ + &\ 5x_2 &\ + &\ 3x_3 & & \\
& \rlap{unter den Nebenbedingungen} & & & & & & & \\
&&  x_1 &\ + &\ 3x_2 &\ + &\  x_3 &\ \leq &\ 3,\ \\
&& -x_1 &\   &\      &\ + &\ 3x_3 &\ \leq &\ 2,\ \\
&& 2x_1 &\ - &\  x_2 &\ + &\ 2x_3 &\ \leq &\ 4,\ \\
&& 2x_1 &\ + &\ 3x_2 &\ - &\  x_3 &\ \leq &\ 2,\ \\
&& & & & & \llap{$x_1, x_2, x_3$} &\ \geq &\ 0.\
\end{alignedat}
\end{align}
Nach der Durchführung des Simplexverfahrens erhält man das folgende letzte Tableau:
\begin{align*}
\begin{alignedat}{5}
\label{eq:2:c5}
x_2 &\ = &\ \frac{ 8}{29} &\ - &\ \frac{ 8}{29}x_7 &\ + &\ \frac{5}{29}x_6 &\ - &\ \frac{ 6}{29}x_5,\ \\[1mm]
x_3 &\ = &\ \frac{30}{29} &\ - &\ \frac{ 1}{29}x_7 &\ - &\ \frac{3}{29}x_6 &\ - &\ \frac{ 8}{29}x_5,\ \\[1mm]
x_1 &\ = &\ \frac{32}{29} &\ - &\ \frac{ 3}{29}x_7 &\ - &\ \frac{9}{29}x_6 &\ + &\ \frac{ 5}{29}x_5,\ \\[1mm]
x_4 &\ = &\ \frac{ 1}{29} &\ + &\ \frac{28}{29}x_7 &\ - &\ \frac{3}{29}x_6 &\ + &\ \frac{21}{29}x_5,\ \\[1mm] \cline{1-9}
  z &\ = &\            10 &\ - &\             2x_7 &\ - &\             x_6 &\ - &\              x_5.\
\end{alignedat}
\end{align*}

\begin{enumerate}[a)]
% Aufgabe P-1 (i)
\item Stellen Sie das zugehörige duale Problem (D) auf.
% Aufgabe P-1 (ii)
\item %Eine optimale Lösung $(x_1^*, x_2^*, x_3^*)$ für $(P)$ haben wir bereits mit dem Simplexverfahren bestimmt.
Lesen Sie eine optimale Lösungen $(x_1^*, x_2^*, x_3^*)$ für (P) und $(y_1^*, y_2^*, y_3^*, y_4^*)$ für (D) am letzten Tableau ab.
% Aufgabe P-1 (iii)
\item Überprüfen Sie, ob die von Ihnen abgelesene Lösung $(y_1^*, y_2^*, y_3^*, y_4^*)$ tatsächlich eine \textit{zulässige} Lösung von (D) ist.
% Aufgabe P-1 (iv)
\item Überprüfen Sie mithilfe des Dualitätssatzes, ob $(y_1^*, y_2^*, y_3^*, y_4^*)$ tatsächlich eine \textit{optimale} Lösung von (D) ist.
% Aufgabe P-1 (v)
\item Bestätigen Sie noch einmal, dass es sich bei $(x_1^*, x_2^*, x_3^*)$ und $(y_1^*, y_2^*, y_3^*, y_4^*)$ um optimale Lösungen von (P) bzw. (D) handelt, indem Sie zeigen, dass die komplementären Schlupfbedingungen (Satz~3 im Skript auf Seite 76) erfüllt sind. 

\end{enumerate}

\subsection*{Hausaufgaben bis zum 16.12.2020 (12:00 Uhr)}
\emph{Bitte reichen Sie Ihre Hausaufgaben in festen Zweier- oder Dreiergruppen bei Moodle ein. Bitte laden Sie ausschließlich \textbf{PDF-Dokumente} hoch, andernfalls können Ihre Hausaufgaben nicht korrigiert werden.}

\Aufgabe[ (duales Problem, 6 Punkte)]{H1}
Wir greifen das Beispiel aus Hausaufgabe H1 a) von Blatt~2 auf und nennen es (P).
\begin{enumerate}[a)]
% Aufgabe H-1a (i)
\item Stellen Sie das zugehörige duale Problem (D) auf.
% Aufgabe H-1a (ii)
\item Eine optimale Lösung $(x_1^*, x_2^*, x_3^*)$ für (P) haben wir bereits mit dem Simplexverfahren bestimmt. Lesen Sie zusätzlich eine optimale Lösung $(y_1^*, y_2^*, y_3^*)$ für (D) am letzten Tableau ab.
% Aufgabe H-1a (iii)
\item Überprüfen Sie, ob die von Ihnen abgelesene Lösung $(y_1^*, y_2^*, y_3^*)$ tatsächlich eine \textit{zulässige} Lösung von (D) ist.
% Aufgabe H-1a (iv)
\item Überprüfen Sie mithilfe des Dualitätssatzes, ob $(y_1^*, y_2^*, y_3^*)$ tatsächlich eine \textit{optimale} Lösung von (D) ist.
% Aufgabe H-1a (v)
\item Bestätigen Sie noch einmal, dass es sich bei $(x_1^*, x_2^*, x_3^*)$ und $(y_1^*, y_2^*, y_3^*)$ um optimale Lösungen von (P) bzw. (D) handelt, indem Sie zeigen, dass die komplementären Schlupfbedingungen (Satz~3 im Skript auf Seite 76) erfüllt sind. 
\end{enumerate}

\Aufgabe[ (Testen auf Optimalität, 4 Punkte)]{H2}
 \begin{enumerate}[a)]
% Aufgabe H-2a
\item Schauen Sie sich die in Abschnitt 7.4 des Skripts im Anschluss an Satz 3' aufgeführten Beispiele 1 und 2 an (Seite 78 f.) und bearbeiten Sie die auf Seite 79 gestellte Aufgabe.

% Aufgabe H-2b
\item Gegeben sei das folgende LP-Problem (P) zusammen mit einer vorgeschlagenen Lösung:
\begin{align*}
\begin{alignedat}{5}
& \text{maximiere } & 3x_1 &\ + &\ 2x_2 & & \\
& \rlap{unter den Nebenbedingungen} & & & & & \\
&&   2x_1 &\ + &\ x_2 &\ \leq &\ 18,\ \\
&&  2x_1 &\ +  &\ 3x_2 &\ \leq &\ 42,\ \\
&&  3x_1 &\ + &\ x_2 &\ \leq &\ 24,\ \\
&& && \llap{$x_1,x_2$} &\ \geq &\ 0.\
\end{alignedat}
\end{align*}

Vorgeschlagene Lösung:
\[
x_1^*= 3, \quad
x_2^*= 12.
\]

Prüfen Sie mithilfe von Satz 3' (im Skript auf Seite 78), ob dies eine optimale Lösung von (P) ist.
\end{enumerate}

\end{document}
