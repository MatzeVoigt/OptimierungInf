\documentclass[a4paper]{article}
\usepackage[ngerman]{babel}
\usepackage{amsmath,amssymb,amsfonts,enumerate,mymacros,url,hyperref}
\usepackage[utf8]{inputenc}
\begin{document}
 \title{\vspace*{-3cm}\"Ubersicht zum Simplexverfahren}
 \author{}
 \date{}
 \maketitle
 Das Ergebnis einer Iteration ist immer ein neues Tableau. Am Ende jeder Iteration wird dieses Ergebnis -- das neue Tableau also -- übersichtlich hingeschrieben -- wobei wir uns an einige sehr \emph{nützliche Konventionen} halten wollen:
 \begin{enumerate}[1)]
  \item Die \emph{$z$-Zeile} wird unten notiert und durch einen Strich abgetrennt.
  \item Im neuen Tableau wird die \emph{Zeile mit der Eingangsvariablen} oben hingeschrieben.
  \item Die Variable, die neu auf der rechten Seite auftaucht (\emph{die Ausgangsvariable}) schließt immer rechts an.
  \item Außerdem ist es für die Übersichtlichkeit wichtig, daß im Tableau \emph{gleiche Variablen immer genau untereinander} geschrieben werden.
  \item \emph{Wahl der Eingangsvariablen}: Kommen mehrere Variablen als Eingangsvariable infrage, so ist es üblich, eine Variable auszuwählen, deren Koeffizient in der $z$-Zeile möglichst groß ist.
 \end{enumerate}
  Wir betrachten das folgende Beispiel, das als \emph{Muster für Übungsaufgaben} dient.\\
  \underline{\textbf{Beispiel:}}
\begin{align*}
\begin{alignedat}{5}
& \text{maximiere } & 5x_1 &\ + &\ 4x_2 &\ + &\ 3x_3 & & \\
& \rlap{unter den Nebenbedingungen} & & & & & & & \\
&&  2x_1 &\ + &\ 3x_2 &\ + &\  x_3 &\ \leq &\ 5,\ \\
&& 4x_1 &\  + &\  x_2  &\ + &\ 2x_3 &\ \leq &\ 11,\ \\
&& 3x_1 &\ + &\  4x_2 &\ + &\ 2x_3 &\ \leq &\ 8,\ \\
&& & & & & \llap{$x_1, x_2, x_3$} &\ \geq &\ 0.\
\end{alignedat}
\end{align*}
\underline{\textbf{Lösung:}} \\ [0.2cm]
\textbf{Starttableau:}
\begin{align*}
\begin{alignedat}{5}
\label{eq:2:c1}
x_4 &\ = &\ 5 &\ - &\  2x_1 &\ - &\ 3x_2 &\ - &\  x_3,\ \\
x_5 &\ = &\ 11 &\ - &\  4x_1 &\ -  &\ x_2   &\ - &\ 2x_3,\ \\
x_6 &\ = &\ 8 &\ - &\ 3x_1 &\ - &\  4x_2 &\ - &\ 2x_3,\ \\ \cline{1-9}
  z &\ = &\   &\   &\ 5x_1 &\ + &\ 4x_2 &\ + &\ 3x_3.\
\end{alignedat}
\end{align*}
\textbf{1. Iteration:}\\
Eingangsvariable: $x_1$, Ausgangsvariable: $x_4$ \\
Es folgt
\begin{align*}
 x_1 &= \tfrac{5}{2} - \tfrac{3}{2}x_2 -\tfrac{1}{2}x_3 - \tfrac{1}{2}x_4, \\
 x_5 &= 11 - 4 \left( \tfrac{5}{2} - \tfrac{3}{2}x_2 - \tfrac{1}{2}x_3 - \tfrac{1}{2}x_4 \right) - x_2 - 2x_3 \\
    &= 1 + 5x_2 + 2x_4, \\
 x_6 &= 8 - 3 \left( \tfrac{5}{2} - \tfrac{3}{2}x_2 - \tfrac{1}{2}x_3 - \tfrac{1}{2}x_4 \right) -4x_2 - 2x_3 \\
    &= \tfrac{1}{2} + \tfrac{1}{2}x_2 - \tfrac{1}{2}x_3 + \tfrac{3}{2}x_4, \\
z   &= 5 \left( \tfrac{5}{2} - \tfrac{3}{2}x_2 - \tfrac{1}{2}x_3 - \tfrac{1}{2}x_4 \right) + 4x_2 + 3x_3 \\
    &= \tfrac{25}{2} - \tfrac{7}{2}x_2 + \tfrac{1}{2}x_3 - \tfrac{5}{2}x_4.
\end{align*}
Ergebnis der 1. Iteration:
\begin{align*}
\begin{alignedat}{5}
x_1 &\ = &\  \tfrac{5}{2} &\ - &\ \tfrac{3}{2}x_2 &\ - &\ \tfrac{1}{2}x_3 &\ - &\ \tfrac{1}{2}x_4,\ \\
x_5 &\ = &\            1 &\ + &\           5x_2 &    &                 &\ + &\           2x_4,\ \\
x_6 &\ = &\  \tfrac{1}{2} &\ + &\ \tfrac{1}{2}x_2 &\ - &\ \tfrac{1}{2}x_3 &\ + &\ \tfrac{3}{2}x_4,\ \\ \cline{1-9}
z   &\ = &\ \tfrac{25}{2} &\ - &\ \tfrac{7}{2}x_2 &\ + &\ \tfrac{1}{2}x_3 &\ - &\ \tfrac{5}{2}x_4.\
\end{alignedat}
\end{align*}
\textbf{2. Iteration:} \\
Eingangsvariable: $x_3$, Ausgangsvariable: $x_6$\\
Es folgt
\begin{align*}
 x_3 &= 1 + x_2 + 3x_4 - 2x_6, \\
 x_1 &= \tfrac{5}{2} - \tfrac{3}{2} - \tfrac{1}{2} \left(1 + x_2 + 3x_4 - 2x_6\right) - \tfrac{1}{2}x_4 \\
    &= 2 - 2x_2 - 2x_4 + x_6, \\
  z &= \tfrac{25}{2} -\tfrac{7}{2}x_2 + \tfrac{1}{2} \left(1 + x_2 + 3x_4 - 2x_6 \right) - \tfrac{5}{2}x_4 \\
    &= 13 - 3x_2 - x_4 - x_6.
\end{align*}
Ergebnis der 2. Iteration:
\begin{align*}
\begin{alignedat}{5}
x_3 &\ = &\  1 &\ + &\  x_2 &\ + &\ 3x_4 &\ - &\ 2x_6,\ \\
x_1 &\ = &\  2 &\ - &\ 2x_2 &\ - &\ 2x_4 &\ + &\  x_6,\ \\
x_5 &\ = &\  1 &\ + &\ 5x_2 &\ + &\ 2x_4 &,    &        \\ \cline{1-9}
z   &\ = &\ 13 &\ - &\ 3x_2 &\ - &\  x_4 &\ - &\  x_6.\
\end{alignedat}
\end{align*}
Dieses Tableau liefert die optimale Lösung $x_1=2$, $x_2 = 0$, $x_3 = 1$ mit $z=13$.\\
\textbf{Hinweis:} \emph{Das Ergebnis einer Iteration ist natürlich auch immer eine neue zulässige Basislösung.} Dies braucht am Ende einer Iteration aber nicht unbedingt hingeschrieben zu werden, da sie implizit im neuen Tableau enthalten und sehr leicht ablesbar ist. \\
Der Vollständigkeit halber geben wir diese zulässigen Basislösungen einmal explizit an:\\
{\textbf{Startlösung:}}
\begin{equation*}
 x_1 = 0, \quad x_2 = 0, \quad x_3 = 0, \quad x_4 = 5, \quad x_5 = 11, \quad x_6 = 8, \quad z = 0 
\end{equation*}
{\textbf{zulässige Basislösung nach der 1. Iteration:}}
\begin{equation*}
 x_1 = \tfrac{5}{2}, \quad x_2 = 0, \quad x_3 = 0, \quad x_4 = 0, \quad x_5 = 1, \quad x_6 = \tfrac{1}{2}, \quad z = 12.5 
\end{equation*}
{\textbf{zulässige Basislösung nach der 2. Iteration:}}
\begin{equation*}
 x_1 = 2, \quad x_2 = 0, \quad x_3 = 1, \quad x_4 = 0, \quad x_5 = 1, \quad x_6 = 0, \quad z = 13 
\end{equation*}
\end{document}

