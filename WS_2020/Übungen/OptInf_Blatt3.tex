\def\nr{3. Aufgabenblatt}
\def\kopf{\\\hfill\normalsize\mdseries}
\documentclass[11pt,a4paper,fleqn]{scrartcl}
\usepackage{eurosym}
%\usepackage{a4kopka}
\usepackage{amsmath,amssymb,amsthm,amsfonts}
\usepackage[utf8]{inputenc}
\usepackage{algorithmic,algorithm}
\usepackage{graphics,graphicx}
\usepackage{pgfplots,tikz}
\usepackage{enumerate}
\usepackage[ngerman]{babel}
% \usepackage[software]{mymacros}
%\usepackage{matrix}
\usepackage{hyperref}
% \usepackage{caption}
\usepackage{caption, subcaption}

\floatname{algorithm}{Algorithmus}
\renewcommand{\algorithmicrequire}{\textbf{Input:}}
\renewcommand{\algorithmicensure}{\textbf{Output:}}

%\usepackage{enumitem} 
%\textheight25cm
\textheight23cm
\topmargin-15mm
\oddsidemargin-5mm    %  -10mm
\textwidth17cm    %   18.8cm
\footskip0pt
\thispagestyle{empty}
\parindent0mm
\parskip0ex
\parskip0ex

\makeatletter
\DeclareOldFontCommand{\rm}{\normalfont\rmfamily}{\mathrm}
\DeclareOldFontCommand{\sf}{\normalfont\sffamily}{\mathsf}
\DeclareOldFontCommand{\tt}{\normalfont\ttfamily}{\mathtt}
\DeclareOldFontCommand{\bf}{\normalfont\bfseries}{\mathbf}
\DeclareOldFontCommand{\it}{\normalfont\itshape}{\mathit}
\DeclareOldFontCommand{\sl}{\normalfont\slshape}{\@nomath\sl}
\DeclareOldFontCommand{\sc}{\normalfont\scshape}{\@nomath\sc}
\makeatother

% \newcommand{\cg}[1]{{\color{blue} #1}}
% \newcommand{\cb}[1]{{\color{green} #1}}
% \newcommand{\cred}[1]{{\color{red} #1}}
% \newcommand{\cc}[1]{{\color{cyan} #1}}
% \newcommand{\cm}[1]{{\color{magenta} #1}}

\newcommand{\Aufgabe}[2][]{\par\bigskip{\sf\bfseries Aufgabe #2#1:}}
%\hspace{3em}{\small(#2 point\ifthenelse{#2>1}{s}{})}}\par\smallskip}
%\newcommand\aufgabe[2][~]{\par\bigskip{\sf\bfseries Aufgabe #1
%    \hspace{3em} \ifthenelse{\equal{#2}{~}}{}{(#2)}}\par\smallskip}
\usepackage{mymacros}

\begin{document}
{\sf Universit\"at Hamburg \hfill Wintersemester 2020/21 \\ Fachbereich Mathematik \\ Dr. Matthias Voigt}
\begin{center}
\ifthenelse{\equal{\nr}{no}}{\Large\sf\bfseries \kopf}{\Large\sf\bfseries Optimierung f\"ur Studierende der Informatik -- \nr.~\kopf}
\end{center}

\renewcommand{\tilde}{\widetilde}
\renewcommand{\hat}{\widehat}
\newcommand{\ri}{\mathrm{i}}
\renewcommand{\H}{\mathsf{H}}
\newcommand{\T}{\mathsf{T}}


\subsection*{Präsenzaufgaben am 23./24.11.2020}

\Aufgabe[ (zulässige Starttableaus)]{P1}
Bestimmen Sie für das LP-Problem ein zulässiges Starttableau bzw. stellen Sie fest, dass das Problem unlösbar ist. Erreichen Sie dies, indem Sie die 1. Phase des Zweiphasen-Simplexverfahrens durchführen.
\begin{align*}
\begin{alignedat}{4}
& \text{maximiere } & 6x_1 &\ + &\ 11x_2 & & \\
& \rlap{unter den Nebenbedingungen} & & & & & \\
&& 4x_1 &\ - &\ x_2 &\ \leq &\  2, \\
&& -x_1 &\ + &\ x_2 &\ \leq &\  8, \\
&& -x_1 &\ - &\ x_2 &\ \leq &\ -3, \\
&&&& \llap{$x_1,x_2$} &\ \geq &\ 0.
\end{alignedat}
\end{align*}

\subsection*{Hausaufgaben bis zum 02.12.2020 (12:00 Uhr)}
\emph{Bitte reichen Sie Ihre Hausaufgaben in festen Zweier- oder Dreiergruppen bei Moodle ein. Bitte laden Sie ausschließlich \textbf{PDF-Dokumente} hoch, andernfalls können Ihre Hausaufgaben nicht korrigiert werden.}

\Aufgabe[ (Lösbarkeit und zulässige Starttableaus, 6 Punkte)]{H1}
Bestimmen Sie für folgende LP-Probleme ein zulässiges Starttableau bzw. stellen Sie fest, dass diese unlösbar sind. Erreichen Sie dies, indem Sie die 1. Phase des Zweiphasen-Simplexverfahrens durchführen.
\begin{enumerate}[a)]
	% Aufgabe H-1a
	\item \begin{align*}
	\begin{alignedat}{4}
	& \text{maximiere } & -10x_1 &\ + &\ 5x_2 & & \\
	& \rlap{unter den Nebenbedingungen} & & & & & \\
	&& -2x_1 &\ - &\ 4x_2 &\ \leq &\ -10, \\
	&&  2x_1 &\ - &\ 2x_2 &\ \leq &\   5, \\
	&&&& \llap{$x_1,x_2$} &\ \geq &\ 0.
	\end{alignedat}
	\end{align*}
	
	% Aufgabe H-2a
	\item \begin{align*}
	\begin{alignedat}{4}
	& \text{maximiere } & -2x_1 &\ + &\ x_2 & & \\
	& \rlap{unter den Nebenbedingungen} & & & & & \\
	&&  x_1 &\ - &\ x_2 &\ \leq &\ -4, \\
	&& 2x_1 &\ + &\ x_2 &\ \leq &\  1, \\
	&& -x_1 &\ - &\ 2x_2 &\ \leq &\ -2, \\
	&&&& \llap{$x_1,x_2$} &\ \geq &\ 0.
	\end{alignedat}
	\end{align*}
\end{enumerate}

\Aufgabe[ (Beispiele von Klee und Minty, 4 Punkte)]{H2}
\begin{enumerate}[a)]
	% Aufgabe H-2a
	\item Schreiben Sie das Problem von Klee und Minty für $n=2$ auf. 
	
	% Aufgabe H-2b
%	\item Stellen Sie die Menge der zulässigen Lösungen dieses Problems durch eine %Skizze dar, wobei Sie den Maßstab wie folgt wählen:
%	\begin{align*}
%	\text{1 Einheit auf der $x_1$-Achse} &\mathrel{\widehat{=}} \text{1cm} \\%
%	\text{10 Einheiten auf der $x_2$-Achse} &\mathrel{\widehat{=}} \text{1cm}.
%	\end{align*}
	
	% Aufgabe H-2c
	\item Lösen Sie das Problem mit dem Simplexverfahren auf zwei verschiedene Arten und stellen Sie für beide Arten fest, wie viele Iterationen benötigt werden.
	\begin{enumerate}[i)]
		% Aufgabe H-2c (i)
		\item Benutzen Sie die Regel vom größten Koeffizienten.
		% Aufgabe H-2c (ii)
		\item Wählen Sie in der 1. Iteration $x_2$ als Eingangsvariable.
    \end{enumerate}
\end{enumerate}

\end{document}
