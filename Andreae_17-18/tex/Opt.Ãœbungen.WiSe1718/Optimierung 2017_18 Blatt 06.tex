\documentclass[11pt, a4paper]{article}
\usepackage{amsmath}
\usepackage{amsfonts}
\usepackage{amssymb}
\usepackage[latin1]{inputenc}
\usepackage[ngerman]{babel}
\usepackage[babel,german=quotes]{csquotes}
\usepackage{fullpage}
\usepackage{paralist}
\usepackage{mathtools}
\usepackage{eurosym}

% Meta Information festlegen
\usepackage{hyperref}
\hypersetup{
  pdftitle={Optimierung f�r Studierende der Informatik, Aufgabenblatt 06},
  pdfauthor={Thomas Andreae},
  pdfcreator={LaTeX2e},
	hyperfootnotes=false,
	breaklinks=true,
	colorlinks=true,
	allcolors={black}}

\setlength{\parindent}{0em}
\pagestyle{empty}

\begin{document}

\begin{center}
\begin{Large}
\textbf{Optimierung f�r Studierende der Informatik}
\end{Large}

\textbf{Thomas Andreae}
	
\vspace{0.5cm}

\textbf{Wintersemester 2017/18}

\textbf{Blatt 6}

\vspace{0.5cm}
\end{center}

\small

\begin{enumerate}[\bfseries A:]

%------------------------------------------------------------------------
% Pr�senzaufgaben
%------------------------------------------------------------------------

\item \textbf{Pr�senzaufgaben am 27./28. November 2017}

\begin{enumerate}[\bfseries 1.]

% Aufgabe P-1
\item Konstruieren Sie das duale Problem:
\begin{enumerate}[a)]
% Aufgabe P-1a
\item Gegeben sei das folgende LP-Problem, das wir $(P)$ nennen wollen:
\begin{align*}
\begin{alignedat}{5}
& \text{maximiere } & x_1 &\ + &\ x_2 &\ + &\ x_3 & & \\
& \rlap{unter den Nebenbedingungen} & & & & & & & \\
&&  2x_1 &\ - &\ 4x_2 &\ + &\  x_3 &\ =    &\ -1\ \\
&&   x_1 &\ + &\ 5x_2 &\ + &\  x_3 &\ =    &\ 16\ \\
&&   x_1 &\   &\      &\ + &\  x_3 &\ \geq &\  5\ \\
&&  2x_1 &\ + &\ 4x_2 &\ - &\  x_3 &\ \leq &\  8\ \\
&&   x_1 &\ - &\ 3x_2 &\ + &\  x_3 &\ \leq &\  0\ \\
&& -4x_1 &\ + &\ 3x_2 &\   &\      &\ \leq &\  4\ \\
&&  4x_1 &\ - &\ 3x_2 &\ + &\ 5x_3 &\ \leq &\ 10\ \\
&&   x_1 &\ + &\ 2x_2 &\ + &\  x_3 &\ \leq &\  9\ \\
&&&&&& \llap{$x_2$} &\ \geq &\ 0.
\end{alignedat}
\end{align*}

Konstruieren Sie das zu $(P)$ duale Problem $(D)$, indem Sie das \textit{Dualisierungsrezept} verwenden.

% Aufgabe P-1b
\item Nun sei mit (P) das folgende Problem bezeichnet:
\begin{align*}
\begin{alignedat}{6}
& \text{minimiere } & x_1 &\ - &\ x_2 & & & & & & \\ 
& \rlap{unter den Nebenbedingungen} & & & & & & & & & \\
&& 2x_1 &\ + &\ 3x_2 &\ - &\  x_3 &\ + &\  x_4 &\ \leq &\ 0\ \\
&& 3x_1 &\ + &\  x_2 &\ + &\ 4x_3 &\ - &\ 2x_4 &\ \geq &\ 3\ \\
&& -x_1 &\ - &\  x_2 &\ + &\ 2x_3 &\ + &\  x_4 &\ =    &\ 1\ \\
&&&&&&&& \llap{$x_2,x_3$} &\ \geq &\ 0.
\end{alignedat}
\end{align*}

Bilden Sie das zu (P) duale Problem, indem Sie das \textit{Dualisierungsrezept} verwenden (diesmal allerdings \enquote{von rechts nach links}).
\end{enumerate}

% Aufgabe P-2
\item In Matrixnotation lautet ein LP-Problem in Standardform bekanntlich so:
\begin{align*}
\begin{alignedat}{3}
& \text{maximiere } & c^Tx & & \\
& \rlap{unter den Nebenbedingungen} & & & \\
&& Ax &\ \leq &\ b\ \\
&& x &\ \geq &\ 0.
\end{alignedat}
\end{align*}

Das Duale hierzu lautet in Matrixnotation:
\begin{align*}
\begin{alignedat}{3}
& \text{minimiere } & b^Ty & & \\
& \rlap{unter den Nebenbedingungen} & & & \\
&& A^Ty &\ \geq &\ c\ \\
&& y &\ \geq &\ 0.
\end{alignedat}
\end{align*}

\pagebreak
Geben Sie das Duale der folgenden beiden Probleme in Matrixnotation an:
\begin{enumerate}[a)]
% Aufgabe P-2a
\item \begin{align*}
\begin{alignedat}{3}
& \text{maximiere } & c^Tx & & \\
& \rlap{unter den Nebenbedingungen} & & & \\
&& Ax &\ \leq &\ b
\end{alignedat}
\end{align*}

% Aufgabe P-2b
\item \begin{align*}
\begin{alignedat}{3}
& \text{maximiere } & c^Tx & & \\
& \rlap{unter den Nebenbedingungen} & & & \\
&& Ax &\ = &\ b \\
&& x &\ \geq &\ 0
\end{alignedat}
\end{align*}
\end{enumerate}

\end{enumerate}

%------------------------------------------------------------------------
% Hausaufgaben
%------------------------------------------------------------------------

\item \textbf{Hausaufgaben zum 4./5. Dezember 2017}

\begin{enumerate}[\bfseries 1.]

% Aufgabe H-1
\item \begin{enumerate}[a)]
% Aufgabe H-1a
\item Gegeben sei das folgende LP-Problem, das wir (P) nennen wollen:
\begin{align*}
\begin{alignedat}{8}
& \text{maximiere } &  x_1 &\ + &\ 2x_2 &\ - &\ 3x_3 & & &\ + &\ x_5 &\ - & x_6 & \\
& \rlap{unter den Nebenbedingungen} & & & & & & & & & & & \\
&& 3x_1 &\ - &\  x_2 &\ + &\ 9x_3 &\ + &\  x_4 &\ - &\  x_5 &\ + &\ 2x_6 &\ \leq &\ -11\ \\
&& -x_1 &\ - &\  x_2 &\ + &\ 2x_3 &\ + &\  x_4 &\ - &\ 2x_5 &    &       &\ \geq &\   4\ \\
&& 7x_1 &\ + &\  x_2 &\ + &\ 4x_3 &\ - &\ 2x_4 &\ + &\  x_5 &    &       &\ =    &\   2\ \\
&& &&&&&&&&&& \llap{$x_2,x_5,x_6$} &\ \geq &\ 0.
\end{alignedat}
\end{align*}

Bilden Sie das zu (P) duale Problem (D), indem Sie das \textit{Dualisierungsrezept} verwenden.

% Aufgabe H-1b
\item Nun sei mit (P) das folgende Problem bezeichnet:
\begin{align*}
\begin{alignedat}{6}
& \text{minimiere } & 5x_1 &\ - &\ 2x_2 &\ + &\ x_3 &\ + &\ x_4 & & \\
& \rlap{unter den Nebenbedingungen} & & & & & & & \\
&&  7x_1 &\ - &\  x_2 &\ + &\  x_3 &\ + &\ 2x_4 &\ \leq &\   5\ \\
&&   x_1 &\ + &\  x_2 &\ + &\  x_3 &    &       &\ =    &\   9\ \\
&&   x_1 &\   &\      &\ + &\  x_3 &    &       &\ \geq &\   6\ \\
&&   x_1 &\ - &\ 3x_2 &\ + &\  x_3 &    &       &\ \leq &\   4\ \\
&&  2x_1 &\ + &\  x_2 &\ - &\  x_3 &\ + &\  x_4 &\ =    &\   8\ \\
&& -4x_1 &\ + &\ 3x_2 &\   &\      &    &       &\ \geq &\   1\ \\
&&   x_1 &\ + &\ 2x_2 &\ + &\  x_3 &\ + &\ 7x_4 &\ \geq &\ -10\ \\
&& &&&&&& \llap{$x_1,x_3$} &\ \geq &\ 0.
\end{alignedat}
\end{align*}

Bilden Sie wieder das zu (P) duale Problem (D), indem Sie das \textit{Dualisierungsrezept} verwenden (diesmal \enquote{von rechts nach links}).

\end{enumerate}

% Aufgabe H-2
\item \begin{enumerate}[a)]

% Aufgabe H-2a
\item Wir greifen das \textit{Kantinenleiterproblem} aus Aufgabe 2a) von Blatt 1 auf. L�sen Sie dieses Problem mithilfe des folgenden Tools:

\begin{center}
\url{http://www.zweigmedia.com/RealWorld/simplex.html}.
\end{center}

% Aufgabe H-2b
\item L�sen Sie mit dem angegebenen Tool auch das \textit{Salatproblem} aus Aufgabe 2b) von Blatt 1.

% Aufgabe H-2c
\item L�sen Sie mit dem angegebenen Tool auch das Problem des Eiscremeherstellers von Blatt 1.

% Aufgabe H-2d
\item L�sen Sie mit dem angegebenen Tool ebenfalls Pauls Di�tproblem (Skript Seite 6).

\end{enumerate}

\end{enumerate}


\end{enumerate}
\end{document}