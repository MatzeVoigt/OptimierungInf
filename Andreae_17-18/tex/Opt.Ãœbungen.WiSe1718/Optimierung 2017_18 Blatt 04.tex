\documentclass[11pt, a4paper]{article}
\usepackage{amsmath}
\usepackage{amsfonts}
\usepackage{amssymb}
\usepackage[latin1]{inputenc}
\usepackage[ngerman]{babel}
\usepackage[babel,german=quotes]{csquotes}
\usepackage{fullpage}
\usepackage{paralist}
\usepackage{pst-all}
\usepackage{eurosym}

% Meta Information festlegen
\usepackage{hyperref}
\hypersetup{
  pdftitle={Optimierung f�r Studierende der Informatik, Aufgabenblatt 04},
  pdfauthor={Thomas Andreae},
  pdfcreator={LaTeX2e},
	hyperfootnotes=false,
	breaklinks=true,
	colorlinks=true,
	allcolors={black}}

\setlength{\parindent}{0em}
\pagestyle{empty}

\begin{document}

\begin{center}
\begin{Large}
\textbf{Optimierung f�r Studierende der Informatik}
\end{Large}

\textbf{Thomas Andreae}
	
\vspace{0.5cm}

\textbf{Wintersemester 2017/18}

\textbf{Blatt 4}

\vspace{0.5cm}
\end{center}

\small

\begin{enumerate}[\bfseries A:]

%------------------------------------------------------------------------
% Pr�senzaufgaben
%------------------------------------------------------------------------

\item \textbf{Pr�senzaufgaben am 13./14. November 2017}

\begin{enumerate}[\bfseries 1.]

% Aufgabe P-1
\item Im Beispiel \enquote{Energieflussproblem} in Abschnitt 6.2 wird erl�utert, wie das Problem eines \textit{Flusses maximaler St�rke} als LP-Problem formuliert werden kann. Beantworten Sie hierzu die folgenden Fragen:
\begin{enumerate}[a)]
% Aufgabe P-1a
\item Wie viele Variablen gibt es in diesem LP-Problem?
% Aufgabe P-1b
\item Wie lautet die Nebenbedingung, die zum Knoten $V_6$ geh�rt?
% Aufgabe P-1c
\item Wie viele Nebenbedingungen gibt es insgesamt? (Die Nichtnegativit�tsbedingungen sollen nicht mitgez�hlt werden.)
% Aufgabe P-1d
\item Wie lautet die Zielfunktion?
\end{enumerate}

% Aufgabe P-2
\item Anstelle des Netzwerks aus 1. betrachten wir nun das folgende Flussnetzwerk ($v_0$ bezeichnet die \textit{Quelle}, $v_3$ die \textit{Senke} und die Zahlen an den Kanten geben die \textit{Kapazit�ten} an):

\begin{center}
\psset{xunit=2.00cm,yunit=1.00cm,linewidth=0.8pt,nodesep=1pt}
\begin{pspicture}(-0.5,-0.5)(2.5,2.5)

\cnode*(0,1){3pt}{V0} \uput{0.25}[180](0,1){$v_0$}
\cnode*(1,2){3pt}{V1} \uput{0.25}[90](1,2){$v_1$}
\cnode*(1,0){3pt}{V2} \uput{0.25}[270](1,0){$v_2$}
\cnode*(2,1){3pt}{V3} \uput{0.25}[0](2,1){$v_3$}

\ncline{->}{V0}{V1} \uput{0.10}[135](0.5, 1.5){$5$}
\ncline{->}{V0}{V2} \uput{0.10}[225](0.5, 0.5){$2$}
\ncline{->}{V1}{V2} \uput{0.10}[  0](1.0, 1.0){$1$}
\ncline{->}{V1}{V3} \uput{0.10}[ 45](1.5, 1.5){$2$}
\ncline{->}{V2}{V3} \uput{0.10}[315](1.5, 0.5){$4$}

\end{pspicture}
\end{center}


Formulieren Sie f�r dieses Netzwerk die Aufgabe, einen Fluss maximaler St�rke zu finden, als ein lineares Programmierungsproblem.

% Aufgabe P-3
\item F�r das Netzwerk aus 2. seien zus�tzlich zu den Kapazit�ten auch noch Kosten f�r jede Kante gegeben:

\begin{center}
\psset{xunit=2.00cm,yunit=1.00cm,linewidth=0.8pt,nodesep=1pt}
\begin{pspicture}(-0.5,-0.5)(2.5,2.5)

\cnode*(0,1){3pt}{V0} \uput{0.25}[180](0,1){$v_0$}
\cnode*(1,2){3pt}{V1} \uput{0.25}[90](1,2){$v_1$}
\cnode*(1,0){3pt}{V2} \uput{0.25}[270](1,0){$v_2$}
\cnode*(2,1){3pt}{V3} \uput{0.25}[0](2,1){$v_3$}

\ncline{->}{V0}{V1} \uput{0.10}[135](0.5, 1.5){$5 \mid 2$}
\ncline{->}{V0}{V2} \uput{0.10}[225](0.5, 0.5){$2 \mid 5$}
\ncline{->}{V1}{V2} \uput{0.10}[  0](1.0, 1.0){$1 \mid 3$}
\ncline{->}{V1}{V3} \uput{0.10}[ 45](1.5, 1.5){$2 \mid 7$}
\ncline{->}{V2}{V3} \uput{0.10}[315](1.5, 0.5){$4 \mid 1$}

\end{pspicture}
\end{center}

Gefragt ist nach einem \textit{kostenminimalen Fluss} der St�rke 4. Formulieren Sie diese Aufgabenstellung als LP-Problem.

\end{enumerate}

%------------------------------------------------------------------------
% Hausaufgaben
%------------------------------------------------------------------------

\item \textbf{Hausaufgaben zum 20./21. November 2017}

\begin{enumerate}[\bfseries 1.]

% Aufgabe H-1
\item \begin{enumerate}[a)]
	% Aufgabe H-1a
	\item Im nachfolgenden Flussnetzwerk bezeichne $v_0$ die Quelle, $v_7$ die Senke und die Zahlen an den Kanten bezeichnen die Kapazit�ten.
	
	\begin{center}
		\psset{xunit=1.0cm,yunit=1.0cm,linewidth=0.8pt,nodesep=1pt}
		\begin{pspicture}(-0.5,-0.5)(6.5,3.5)
		\footnotesize
		
		\cnode*(0,1.5){3pt}{V0} \uput{0.2}[180](0,1.5){$v_0$}
		\cnode*(2,0.0){3pt}{V1} \uput{0.2}[270](2,0.0){$v_1$}
		\cnode*(2,1.0){3pt}{V2} \uput{0.2}[ 90](2,1.0){$v_2$}
		\cnode*(2,2.0){3pt}{V3} \uput{0.2}[270](2,2.0){$v_3$}
		\cnode*(2,3.0){3pt}{V4} \uput{0.2}[ 90](2,3.0){$v_4$}
		\cnode*(4,1.5){3pt}{V5} \uput{0.2}[  0](4,1.5){$v_5$}
		\cnode*(6,0.0){3pt}{V6} \uput{0.2}[270](6,0.0){$v_6$}
		\cnode*(6,3.0){3pt}{V7} \uput{0.2}[ 90](6,3.0){$v_7$}
		
		
		\ncline{->}{V0}{V1} \uput{0.1}[270](1.0, 0.75){$6$}
		\ncline{->}{V0}{V2} \uput{0.1}[270](1.0, 1.25){$1$}
		\ncline{->}{V0}{V3} \uput{0.1}[ 90](1.0, 1.75){$3$}
		\ncline{->}{V0}{V4} \uput{0.1}[ 90](1.0, 2.25){$5$}
		\ncline{->}{V1}{V6} \uput{0.1}[270](4.0, 0.00){$4$}
		\ncline{->}{V2}{V1} \uput{0.1}[  0](2.0, 0.50){$4$}
		\ncline{->}{V2}{V5} \uput{0.1}[270](3.0, 1.25){$5$}
		\ncline{->}{V2}{V6} \uput{0.1}[270](4.0, 0.50){$2$}
		\ncline{->}{V3}{V4} \uput{0.1}[  0](2.0, 2.50){$5$}
		\ncline{->}{V3}{V5} \uput{0.1}[ 90](3.0, 1.75){$4$}
		\ncline{->}{V4}{V7} \uput{0.1}[ 90](4.0, 3.00){$8$}
		\ncline{->}{V5}{V7} \uput{0.1}[270](5.0, 0.75){$8$}
		\ncline{->}{V6}{V5} \uput{0.1}[ 90](5.0, 2.25){$2$}
		\ncline{->}{V6}{V7} \uput{0.1}[  0](6.0, 1.50){$3$}
		
		\normalsize
		\end{pspicture}
	\end{center}
	
	Formulieren Sie f�r dieses Netzwerk die Aufgabe, einen Fluss maximaler St�rke zu finden, als ein lineares Programmierungsproblem.
	
	% Aufgabe H-1b
	\item F�r das folgende Netzwerk mit Quelle $v_0$ und Senke $v_6$ seien neben den Kapazit�ten auch noch Kosten gegeben; die linke Zahl bezeichne die Kapazit�t, die rechte die Kosten einer Kante:
	
	\begin{center}
		\psset{xunit=1.50cm,yunit=1.25cm,linewidth=0.8pt,nodesep=1pt}
		\begin{pspicture}(-0.5,-1)(3.25,3)
		\footnotesize
		
		\cnode*(0,1){3pt}{V0} \uput{0.25}[180](0,1){$v_0$}
		\cnode*(1,2){3pt}{V1} \uput{0.15}[315](1,2){$v_1$}
		\cnode*(1,1){3pt}{V2} \uput{0.25}[225](1,1){$v_2$}
		\cnode*(1,0){3pt}{V3} \uput{0.15}[ 45](1,0){$v_3$}
		\cnode*(2,1){3pt}{V4} \uput{0.25}[270](2,1){$v_4$}
		\cnode*(3,0){3pt}{V5} \uput{0.25}[270](3,0){$v_5$}
		\cnode*(3,2){3pt}{V6} \uput{0.25}[ 90](3,2){$v_6$}
		
		\ncline{->}{V0}{V1} \uput{0.05}[135](0.5, 1.5){$3 \mid 3$}
		\ncline{->}{V0}{V2} \uput{0.05}[ 90](0.5, 1.0){$4 \mid 4$}
		\ncline{->}{V0}{V3} \uput{0.05}[225](0.5, 0.5){$4 \mid 3$}
		\nccurve[angleA=225,angleB=225]{->}{V0}{V5} \uput{0.05}[270](2, -0.5){$4 \mid 3$}
		\nccurve[angleA=135,angleB=135]{->}{V0}{V6} \uput{0.05}[ 90](2,  2.5){$5 \mid 2$}
		\ncline{->}{V1}{V2} \uput{0.05}[  0](1.0, 1.5){$6 \mid 6$}
		\ncline{->}{V1}{V6} \uput{0.05}[ 90](2.0, 2.0){$1 \mid 3$}
		\ncline{->}{V2}{V3} \uput{0.05}[  0](1.0, 0.5){$1 \mid 6$}
		\ncline{->}{V2}{V4} \uput{0.05}[ 90](1.5, 1.0){$4 \mid 5$}
		\ncline{->}{V3}{V5} \uput{0.05}[270](2.0, 0.0){$2 \mid 5$}
		\ncline{->}{V4}{V5} \uput{0.05}[225](2.5, 0.5){$1 \mid 3$}
		\ncline{->}{V4}{V6} \uput{0.05}[135](2.5, 1.5){$7 \mid 4$}
		\ncline{->}{V5}{V6} \uput{0.05}[  0](3.0, 1.0){$4 \mid 2$}
		
		\normalsize
		\end{pspicture}
	\end{center}
	
	Gefragt ist nach einem kostenminimalen Fluss der St�rke 6. Formulieren Sie diese Aufgabenstellung als LP-Problem.
\end{enumerate}


% Aufgabe H-2
\item 
\begin{enumerate}[a)]
	
	% Aufgabe H-2a
	\item Ein Personalchef habe f�r 3 offene Stellen 5 Bewerber, wobei aufgrund eines Eignungstests bekannt sei, welche Einarbeitungszeit $t_{ij}$ der Bewerber $i$ f�r die Stelle $j$ ben�tigt ($i=1,\ldots,5$ und $j=1,2,3$). Es sollen alle drei Stellen besetzt werden -- zwei Bewerber gehen leer aus. Die Einstellung soll so erfolgen, dass die Summe der Einarbeitungszeiten minimal ist. Formulieren Sie diese Aufgabe als ein bin�res LP-Problem und veranschaulichen Sie die Fragestellung mithilfe eines bipartiten Graphen.
	
	% Aufgabe H-2b
	\item Es sei $G=(V,E)$ ein ungerichteter Graph mit Knotenmenge $V = \bigl\{ v_1,\ldots,v_n \bigr\}$ und Kantenmenge $E = \bigl\{ e_1,\ldots, e_m \bigr\}$. Eine Teilmenge $U$ der Knotenmenge hei�t \textit{unabh�ngig}, falls keine zwei Knoten von $U$ durch eine Kante verbunden sind. Ein bekanntes, in der Informatik h�ufig betrachtetes Optimierungsproblem:
	\begin{equation}
	\tag{$\star$}
	\text{Finde in $G$ eine unabh�ngige Menge $U$, die m�glichst viele Knoten enth�lt.}
	\end{equation}
	
	Formulieren Sie das Problem ($\star$) f�r den unten abgebildeten Graphen $G$ als ein bin�res LP-Problem.
	
	\begin{center}
		\psset{xunit=1.00cm,yunit=1.00cm,linewidth=0.8pt,nodesep=1pt}
		\begin{pspicture}(-0.5,-0.5)(4,2.5)
		
		\cnode*(0,1){3pt}{V1} \uput{0.25}[270](0,1){$v_1$}
		\cnode*(2,2){3pt}{V2} \uput{0.25}[ 90](2,2){$v_2$}
		\cnode*(2,0){3pt}{V3} \uput{0.25}[270](2,0){$v_3$}
		\cnode*(4,2){3pt}{V4} \uput{0.25}[ 90](4,2){$v_4$}
		\cnode*(4,0){3pt}{V5} \uput{0.25}[270](4,0){$v_5$}
		\cnode*(6,1){3pt}{V6} \uput{0.25}[  0](6,1){$v_6$}
		
		\ncline{-}{V1}{V2}
		\ncline{-}{V1}{V3}
		\ncline{-}{V2}{V3}
		\ncline{-}{V2}{V4}
		\ncline{-}{V3}{V5}
		\ncline{-}{V4}{V5}
		\ncline{-}{V4}{V6}
		\ncline{-}{V5}{V6}
		
		\end{pspicture}
	\end{center}
	
	\textbf{Hinweis}: F�r jeden Knoten $v_i$ ist eine Variable $x_i$ zu betrachten und f�r jede Kante ist eine Nebenbedingung zu formulieren.
	
	
	% Aufgabe H-2c
	\item Formulieren Sie das bin�re Problem aus b) in ein Ganzzahliges Lineares Programmierungsproblem (ILP-Problem) um. Wie lautet die LP-Relaxation dieses Problems?
\end{enumerate}
\end{enumerate}

\end{enumerate}
\end{document}