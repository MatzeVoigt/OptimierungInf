\documentclass[11pt, a4paper]{article}
\usepackage{amsmath}
\usepackage{amsfonts}
\usepackage{amssymb}
\usepackage[latin1]{inputenc}
\usepackage[ngerman]{babel}
\usepackage[babel,german=quotes]{csquotes}
\usepackage{fullpage}
\usepackage{paralist}
\usepackage{mathtools}

% Meta Information festlegen
\usepackage{hyperref}
\hypersetup{
  pdftitle={Optimierung f�r Studierende der Informatik, Aufgabenblatt 03},
  pdfauthor={Thomas Andreae},
  pdfcreator={LaTeX2e},
	hyperfootnotes=false,
	breaklinks=true,
	colorlinks=true,
	allcolors={black}}

\setlength{\parindent}{0em}
\pagestyle{empty}

\begin{document}

\begin{center}
\begin{Large}
\textbf{Optimierung f�r Studierende der Informatik}
\end{Large}

\textbf{Thomas Andreae}
	
\vspace{0.5cm}

\textbf{Wintersemester 2017/18}

\textbf{Blatt 3}

\vspace{0.5cm}
\end{center}

\small

\begin{enumerate}[\bfseries A:]

%------------------------------------------------------------------------
% Pr�senzaufgaben
%------------------------------------------------------------------------

\item \textbf{Pr�senzaufgaben am 6./7. November 2017}

\begin{enumerate}[\bfseries 1.]


% Aufgabe P-1
\item Bestimmen Sie f�r das LP-Problem ein zul�ssiges Starttableau bzw. stellen Sie fest, dass das Problem unl�sbar ist. Erreichen Sie dies, indem Sie die 1. Phase des Zweiphasen-Simplexverfahrens durchf�hren.
\begin{align*}
\begin{alignedat}{4}
& \text{maximiere } & 6x_1 &\ + &\ 11x_2 & & \\
& \rlap{unter den Nebenbedingungen} & & & & & \\
&& 4x_1 &\ - &\ x_2 &\ \leq &\  2 \\
&& -x_1 &\ + &\ x_2 &\ \leq &\  8 \\
&& -x_1 &\ - &\ x_2 &\ \leq &\ -3 \\
&&&& \llap{$x_1,x_2$} &\ \geq &\ 0
\end{alignedat}
\end{align*}	
	
\end{enumerate}

%------------------------------------------------------------------------
% Hausaufgaben
%------------------------------------------------------------------------

\item \textbf{Hausaufgaben zum 13./14. November 2017}

\begin{enumerate}[\bfseries 1.]

% Aufgabe H-1
\item Bestimmen Sie f�r das LP-Problem ein zul�ssiges Starttableau bzw. stellen Sie fest, dass das Problem unl�sbar ist. Erreichen Sie dies, indem Sie die 1. Phase des Zweiphasen-Simplexverfahrens durchf�hren.
\begin{enumerate}[a)]
	% Aufgabe H-1a
	\item \begin{align*}
	\begin{alignedat}{4}
	& \text{maximiere } & -7x_1 &\ + &\ 10x_2 & & \\
	& \rlap{unter den Nebenbedingungen} & & & & & \\
	&& -x_1 &\ - &\ 2x_2 &\ \leq &\ -12 \\
	&&  x_1 &\ - &\ 4x_2 &\ \leq &\   7 \\
	&&&& \llap{$x_1,x_2$} &\ \geq &\ 0
	\end{alignedat}
	\end{align*}
	
	% Aufgabe H-2a
	\item \begin{align*}
	\begin{alignedat}{4}
	& \text{maximiere } & -13x_1 &\ + &\ 5x_2 & & \\
	& \rlap{unter den Nebenbedingungen} & & & & & \\
	&&  x_1 &\ - &\ x_2 &\ \leq &\ -1 \\
	&& 2x_1 &\ + &\ x_2 &\ \leq &\  2 \\
	&& -x_1 &\ - &\ x_2 &\ \leq &\ -4 \\
	&&&& \llap{$x_1,x_2$} &\ \geq &\ 0
	\end{alignedat}
	\end{align*}
\end{enumerate}

% Aufgabe H-2
\item \begin{enumerate}[a)]
	% Aufgabe H-2a
	\item Schreiben Sie das Klee-Minty Problem f�r $n=2$ auf. 
	
	% Aufgabe H-2b
	\item Stellen Sie die Menge der zul�ssigen L�sungen dieses Problems durch eine Skizze dar, wobei Sie den Ma�stab wie folgt w�hlen:
	\begin{align*}
	\text{1 Einheit auf der $x_1$-Achse} &\mathrel{\widehat{=}} \text{1cm} \\
	\text{10 Einheiten auf der $x_2$-Achse} &\mathrel{\widehat{=}} \text{1cm}.
	\end{align*}
	
	% Aufgabe H-2c
	\item L�sen Sie das Problem mit dem Simplexverfahren auf zwei verschiedene Arten und stellen Sie f�r beide Arten fest, wie viele Iterationen ben�tigt werden.
	\begin{enumerate}[ (i)]
		% Aufgabe H-2c (i)
		\item Benutzen Sie die Regel vom gr��ten Koeffizienten.
		% Aufgabe H-2c (ii)
		\item W�hlen Sie in der 1. Iteration $x_2$ als Eingangsvariable.
	\end{enumerate}
\end{enumerate}


\end{enumerate}
\end{enumerate}
\end{document}