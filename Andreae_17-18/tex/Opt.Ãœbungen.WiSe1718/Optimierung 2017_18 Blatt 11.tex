\documentclass[11pt, a4paper]{article}
\usepackage{amsmath}
\usepackage{amsfonts}
\usepackage{amssymb}
\usepackage[latin1]{inputenc}
\usepackage[ngerman]{babel}
\usepackage[babel,german=quotes]{csquotes}
\usepackage{fullpage}
\usepackage{paralist}
\usepackage{pst-all}

% Meta Information festlegen
\usepackage{hyperref}
\hypersetup{
  pdftitle={Optimierung f�r Studierende der Informatik, Aufgabenblatt 11},
  pdfauthor={Thomas Andreae},
  pdfcreator={LaTeX2e},
	hyperfootnotes=false,
	breaklinks=true,
	colorlinks=true,
	allcolors={black}}

\setlength{\parindent}{0em}
\pagestyle{empty}

\begin{document}

\begin{center}
\begin{Large}
\textbf{Optimierung f�r Studierende der Informatik}
\end{Large}

\textbf{Thomas Andreae}
	
\vspace{0.5cm}

\textbf{Wintersemester 2017/18}

\textbf{Blatt 11}

\vspace{0.5cm}
\end{center}

\small

\begin{enumerate}[\bfseries A:]

%------------------------------------------------------------------------
% Pr�senzaufgaben
%------------------------------------------------------------------------

\item \textbf{Pr�senzaufgaben am 15./16. Januar 2018}

\begin{enumerate}[\bfseries 1.]


% Aufgabe P-1
\item Gegeben sei eine Menge $S = \bigl\{ s_1,\ldots,s_n \bigr\}$ und eine Kollektion $T_1,\ldots,T_m$ von $k$-elementigen Teilmengen von $S$. Au�erdem besitze jedes Element $s_i$ ein Gewicht $w_i \geq 0$ mit $w_i \in \mathbb{Q}$ ($i=1,\ldots,n$).

Zur Erinnerung: Eine Teilmenge $H \subseteq S$ wird ein \textit{Hitting Set} genannt, falls $H \cap T_i \neq \emptyset$ f�r alle $i=1,\ldots,m$ gilt. Gesucht ist ein Hitting Set $H$, dessen Gewicht so klein wie m�glich ist. Anders gesagt: Die Summe
\[
\sum\limits_{s_i \in H}{w_i}
\]
soll so klein wie m�glich sein. Wir wollen das beschriebene Problem WEIGHTED $k$-HITTING SET nennen.

\begin{enumerate}[a)]
% Aufgabe P-1a
\item Formulieren Sie dieses Problem als ein ganzzahliges Programmierungsproblem, dass Sie (ILP) nennen.
% Aufgabe P-1b
\item Wie lautet die LP-Relaxation (LP) dieses Problems?
\end{enumerate}



% Aufgabe P-2
\item Der Graph $G = (V,E)$ mit L�ngenfunktion $\ell$ sei durch die folgende Zeichnung gegeben:

\begin{center}
\psset{xunit=1.00cm,yunit=1.00cm,linewidth=0.8pt,nodesep=0.5pt}
\begin{pspicture}(-0.5,-0.5)(6.5,4.5)

\cnode*(2,4){3pt}{A} \uput{0.25}[ 90](2,4){$a$}
\cnode*(2,2){3pt}{B} \uput{0.25}[225](2,2){$b$}
\cnode*(2,0){3pt}{C} \uput{0.25}[270](2,0){$c$}
\cnode*(6,4){3pt}{D} \uput{0.25}[ 90](6,4){$d$}
\cnode*(6,2){3pt}{E} \uput{0.25}[  0](6,2){$e$}
\cnode*(6,0){3pt}{F} \uput{0.25}[270](6,0){$f$}
\cnode*(0,2){3pt}{S} \uput{0.25}[180](0,2){$s$}

\ncline{->}{S}{A} \uput{0.10}[ 90](1.0, 3.0){$6$}
\ncline{->}{S}{B} \uput{0.10}[ 90](1.0, 2.0){$5$}
\ncline{->}{S}{C} \uput{0.10}[270](1.0, 1.0){$2$}
\ncarc[arcangle=330]{->}{A}{B} \uput{0.35}[180](2.0, 3.0){$1$}
\ncline{->}{A}{D} \uput{0.10}[ 90](4.0, 4.0){$2$}
%\ncline{->}{A}{E} \uput{0.10}[ 90](3.0, 3.5){$3$}
\ncarc[arcangle=345]{->}{A}{E} \uput{0.20}[45](3.0, 3.0){$3$}
\ncarc[arcangle=330]{->}{B}{A} \uput{0.35}[  0](2.0, 3.0){$3$}
\ncline{->}{B}{D} \uput{0.10}[ 90](3.0, 2.5){$4$}
\ncline{->}{B}{E} \uput{0.10}[ 90](4.0, 2.0){$5$}
\ncline{->}{B}{F} \uput{0.05}[ 90](3.0, 1.5){$3$}
\ncline{->}{C}{B} \uput{0.10}[  0](2.0, 1.0){$2$}
%\ncline{->}{C}{E} \uput{0.10}[ 90](3.0, 0.5){$1$}
\ncarc[arcangle=15]{->}{C}{E} \uput{0.20}[315](3.0, 1.0){$1$}
\ncline{->}{C}{F} \uput{0.10}[270](4.0, 0.0){$3$}
\ncline{->}{D}{E} \uput{0.10}[  0](6.0, 3.0){$2$}
\ncarc[arcangle=345]{->}{E}{A} \uput{0.10}[ 45](3.5, 3.5){$2$}
\ncarc[arcangle=15]{->}{E}{C} \uput{0.10}[315](3.5, 0.5){$2$}
\ncarc[arcangle=330]{->}{E}{F} \uput{0.35}[180](6.0, 1.0){$1$}
\ncarc[arcangle=330]{->}{F}{E} \uput{0.35}[  0](6.0, 1.0){$3$}

\end{pspicture}
\end{center}



\begin{enumerate}[a)]
% Aufgabe P-2a
\item Verwenden Sie den Algorithmus von Dijkstra in der Version auf Seite 181 des Skripts, um f�r alle $v \in V$ die L�nge $d(v)$ eines k�rzesten $s,v$-Pfades zu berechnen. Legen Sie eine Tabelle wie auf Seite 183 des Skripts an, d.h., notieren Sie auch immer einen \enquote{Vorg�ngerknoten}.

% Aufgabe P-2b
\item Bestimmen Sie einen k�rzeste-Pfade-Baum anhand der Eintr�ge in der letzten Zeile Ihrer Tabelle.
\end{enumerate}

\end{enumerate}

%------------------------------------------------------------------------
% Hausaufgaben
%------------------------------------------------------------------------

\item \textbf{Hausaufgaben zum 22./23. Januar 2018}

\begin{enumerate}[\bfseries 1.]


% Aufgabe H-1
\item Wir kn�pfen an Pr�senzaufgabe 1 an und betrachten das dort formulierte Problem WEIGHTED $k$-HITTING SET. Die Bezeichnungen (ILP) und (LP) verwenden wir wie in dieser Pr�senzaufgabe.
\begin{enumerate}[a)]
% Aufgabe H-1a
\item Geben Sie basierend auf (LP) einen (polynomiellen) Approximationsalgorithmus f�r WEIGHTED $k$-HITTING SET an, bei dem es sich um einen $k$-Approximationsalgorithmus handelt.
% Aufgabe H-1b
\item Weisen Sie nach, dass es sich bei dem von Ihnen angegebenen Algorithmus tats�chlich um einen $k$-Approximationsalgorithmus handelt. 
\end{enumerate}

\pagebreak

% Aufgabe H-2
\item \begin{enumerate}[a)]
% Aufgabe H-2a
\item Der Graph $G=(V,E)$ mit L�ngenfunktion $\ell$ sei durch die folgende Zeichnung gegeben:

\begin{center}
\psset{xunit=1.00cm,yunit=1.00cm,linewidth=0.8pt,nodesep=0.5pt}
\begin{pspicture}(-0.5,-0.75)(6.5,2.75)

\cnode*(0,2){3pt}{S} \uput{0.25}[ 90](0,2){$s$}
\cnode*(2,2){3pt}{C} \uput{0.25}[ 90](2,2){$c$}
\cnode*(4,2){3pt}{D} \uput{0.25}[ 90](4,2){$d$}
\cnode*(6,2){3pt}{G} \uput{0.25}[ 90](6,2){$g$}
\cnode*(0,0){3pt}{A} \uput{0.25}[270](0,0){$a$}
\cnode*(2,0){3pt}{B} \uput{0.25}[270](2,0){$b$}
\cnode*(4,0){3pt}{E} \uput{0.25}[270](4,0){$e$}
\cnode*(6,0){3pt}{F} \uput{0.25}[270](6,0){$f$}


\ncline{->}{S}{A} \uput{0.10}[180](0.0, 1.0){$4$}
\ncline{->}{S}{B} \uput{0.05}[225](1.0, 1.0){$7$}
\ncline{->}{S}{C} \uput{0.10}[ 90](1.0, 2.0){$2$}
\ncline{->}{A}{B} \uput{0.10}[270](1.0, 0.0){$5$}
\ncline{->}{C}{B} \uput{0.10}[180](2.0, 1.0){$6$}
\ncline{->}{C}{D} \uput{0.10}[ 90](3.0, 2.0){$2$}
\ncline{->}{C}{E} \uput{0.05}[225](3.0, 1.0){$6$}
\ncline{->}{D}{E} \uput{0.10}[  0](4.0, 1.0){$2$}
\ncline{->}{D}{G} \uput{0.10}[ 90](5.0, 2.0){$1$}
\ncline{->}{E}{B} \uput{0.10}[270](3.0, 0.0){$1$}
\ncline{->}{E}{F} \uput{0.10}[270](5.0, 0.0){$1$}
\ncline{->}{G}{E} \uput{0.05}[315](5.0, 1.0){$1$}
\ncline{->}{G}{F} \uput{0.10}[  0](6.0, 1.0){$4$}

\end{pspicture}
\end{center}

Verwenden Sie den Algorithmus von Dijkstra (Skript, Seite 181) um f�r alle $v \in V$ die L�nge $d(v)$ eines k�rzesten $s,v$-Pfades zu berechnen. Legen Sie eine Tabelle an, an der man zus�tzlich k�rzeste $s,v$-Pfade ablesen kann. Bestimmen Sie auch einen k�rzeste-Pfade-Baum. 


% Aufgabe H-2b
\item F�r den folgenden Graphen bestimme man einen minimalen aufspannenden Baum auf drei Arten:
\begin{enumerate}[(i)]
\item mit dem Algorithmus von Prim (mit Startknoten $a$);
\item mit dem Algorithmus von Kruskal;
\item mit dem Reverse-Delete-Algorithmus.
\end{enumerate}

Geben Sie jeweils die Kanten in der Reihenfolge an, in der sie hinzugef�gt bzw. weggelassen wurden. (Kommen mehrere Kanten infrage, so w�hle man willk�rlich eine aus.)

\begin{center}
\psset{xunit=1.00cm,yunit=1.00cm,linewidth=0.8pt,nodesep=0.5pt}
\begin{pspicture}(-0.5,-0.75)(10.5,2.75)

\cnode*( 0,0){3pt}{A} \uput{0.25}[270]( 0,0){$a$}
\cnode*( 2,0){3pt}{B} \uput{0.25}[270]( 2,0){$b$}
\cnode*( 4,0){3pt}{C} \uput{0.25}[270]( 4,0){$c$}
\cnode*( 6,0){3pt}{D} \uput{0.25}[270]( 6,0){$d$}
\cnode*( 8,0){3pt}{E} \uput{0.25}[270]( 8,0){$e$}
\cnode*(10,0){3pt}{F} \uput{0.25}[270](10,0){$f$}
\cnode*(10,2){3pt}{G} \uput{0.25}[ 90](10,2){$g$}
\cnode*( 8,2){3pt}{H} \uput{0.25}[ 90]( 8,2){$h$}
\cnode*( 6,2){3pt}{I} \uput{0.25}[ 90]( 6,2){$i$}
\cnode*( 4,2){3pt}{J} \uput{0.25}[ 90]( 4,2){$j$}
\cnode*( 2,2){3pt}{K} \uput{0.25}[ 90]( 2,2){$k$}
\cnode*( 0,2){3pt}{L} \uput{0.25}[ 90]( 0,2){$\ell$}

\ncline{-}{A}{B} \uput{0.10}[270]( 1.0, 0.0){$4$}
\ncline{-}{A}{L} \uput{0.10}[180]( 0.0, 1.0){$6$}
\ncline{-}{B}{C} \uput{0.10}[270]( 3.0, 0.0){$1$}
\ncline{-}{B}{K} \uput{0.10}[180]( 2.0, 1.0){$5$}
\ncline{-}{B}{L} \uput{0.05}[225]( 1.0, 1.0){$8$}
\ncline{-}{C}{D} \uput{0.10}[270]( 5.0, 0.0){$2$}
\ncline{-}{C}{I} \uput{0.05}[315]( 5.0, 1.0){$1$}
\ncline{-}{C}{J} \uput{0.10}[180]( 4.0, 1.0){$2$}
\ncline{-}{C}{K} \uput{0.05}[225]( 3.0, 1.0){$6$}
\ncline{-}{D}{E} \uput{0.10}[270]( 7.0, 0.0){$2$}
\ncline{-}{D}{H} \uput{0.05}[315]( 7.0, 1.0){$5$}
\ncline{-}{D}{I} \uput{0.05}[  0]( 6.0, 1.0){$4$}
\ncline{-}{E}{F} \uput{0.10}[270]( 9.0, 0.0){$4$}
\ncline{-}{E}{G} \uput{0.05}[315]( 9.0, 1.0){$3$}
\ncline{-}{E}{H} \uput{0.10}[  0]( 8.0, 1.0){$3$}
\ncline{-}{F}{G} \uput{0.10}[  0](10.0, 1.0){$7$}
\ncline{-}{G}{H} \uput{0.10}[ 90]( 9.0, 2.0){$1$}
\ncline{-}{H}{I} \uput{0.10}[ 90]( 7.0, 2.0){$3$}
\ncline{-}{I}{J} \uput{0.10}[ 90]( 5.0, 2.0){$3$}
\ncline{-}{J}{K} \uput{0.10}[ 90]( 3.0, 2.0){$2$}
\ncline{-}{K}{L} \uput{0.10}[ 90]( 1.0, 2.0){$1$}

\end{pspicture}
\end{center}

\end{enumerate}


\end{enumerate}
\end{enumerate}
\end{document}