\documentclass[11pt, a4paper]{article}
\usepackage{amsmath}
\usepackage{amsfonts}
\usepackage{amssymb}
\usepackage[latin1]{inputenc}
\usepackage[ngerman]{babel}
\usepackage[babel,german=quotes]{csquotes}
\usepackage{fullpage}
\usepackage{paralist}
\usepackage{mathtools}

% Meta Information festlegen
\usepackage{hyperref}
\hypersetup{
  pdftitle={Optimierung f�r Studierende der Informatik, Aufgabenblatt 07},
  pdfauthor={Thomas Andreae},
  pdfcreator={LaTeX2e},
	hyperfootnotes=false,
	breaklinks=true,
	colorlinks=true,
	allcolors={black}}

\setlength{\parindent}{0em}
\pagestyle{empty}

\begin{document}

\begin{center}
\begin{Large}
\textbf{Optimierung f�r Studierende der Informatik}
\end{Large}

\textbf{Thomas Andreae}
	
\vspace{0.5cm}

\textbf{Wintersemester 2017/18}

\textbf{Blatt 7}

\vspace{0.5cm}
\end{center}

\small

\begin{enumerate}[\bfseries A:]

%------------------------------------------------------------------------
% Pr�senzaufgaben
%------------------------------------------------------------------------

\item \textbf{Pr�senzaufgaben am 4./5. Dezember 2017}

\begin{enumerate}[\bfseries 1.]

% Aufgabe P-1
\item L�sen Sie das folgende LP-Problem mit dem revidierten Simplexverfahren:
\begin{align*}
\begin{alignedat}{5}
& \text{maximiere } & x_1 &\  &\  &\ - &\ 2x_3 & & \\
& \rlap{unter den Nebenbedingungen} & & & & & & & \\
&& -x_1 &\ - &\ 2x_2 &\ + &\ 3x_3 &\ \leq &\ 2\ \\
&& 2x_1 &\ + &\  x_2 &\   &\      &\ \leq &\ 5\ \\
&&  x_1 &\ - &\  x_2 &\ - &\ 3x_3 &\ \leq &\ 4\ \\
&&&&&& \llap{$x_1,x_2,x_3$} &\ \geq &\ 0.
\end{alignedat}
\end{align*}

\end{enumerate}

%------------------------------------------------------------------------
% Hausaufgaben
%------------------------------------------------------------------------

\item \textbf{Hausaufgaben zum 11./12. Dezember 2017}

\begin{enumerate}[\bfseries 1.]

% Aufgabe H-1
\item L�sen Sie das folgende LP-Problem mit dem revidierten Simplexverfahren:
\begin{align*}
\begin{alignedat}{5}
& \text{maximiere } & 2x_1 &\ + &\ 3x_2 &\ + &\ 2x_3 & & \\
& \rlap{unter den Nebenbedingungen} & & & & & & & \\
&&  x_1 &\ + &\ x_2 &\   &\      &\ \leq &\  8\ \\
&&      &\   &\ x_2 &\ + &\ 2x_3 &\ \leq &\ 12\ \\
&&      &\   &\ x_2 &\ + &\  x_3 &\ \leq &\  7\ \\
&&&&&& \llap{$x_1,x_2,x_3$} &\ \geq &\ 0.
\end{alignedat}
\end{align*}

% Aufgabe H2
\item L�sen Sie das folgende LP-Problem mit dem revidierten Simplexverfahren:
\begin{align*}
\begin{alignedat}{6}
& \text{maximiere } & 6x_1 &\ - &\ 9x_2 &\ + &\ x_3 &\ - &\ 11x_4 & & \\
& \rlap{unter den Nebenbedingungen} & & & & & & & & & \\
&& 2x_1 &\ - &\ 3x_2 &\ - &\ x_3 &\ - &\ 7x_4 &\ \leq &\  1\ \\
&& 2x_1 &\ + &\  x_2 &\ + &\ x_3 &\ + &\ 3x_4 &\ \leq &\  3\ \\
&&&&&&&& \llap{$x_1,x_2,x_3,x_4$} &\ \geq &\ 0.
\end{alignedat}
\end{align*}

\end{enumerate}


\end{enumerate}
\end{document}