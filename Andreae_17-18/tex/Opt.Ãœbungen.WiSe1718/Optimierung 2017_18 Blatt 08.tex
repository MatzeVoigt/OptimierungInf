\documentclass[11pt, a4paper]{article}
\usepackage{amsmath}
\usepackage{amsfonts}
\usepackage{amssymb}
\usepackage[latin1]{inputenc}
\usepackage[ngerman]{babel}
\usepackage[babel,german=quotes]{csquotes}
\usepackage{fullpage}
\usepackage{paralist}
\usepackage{pst-all}

% Meta Information festlegen
\usepackage{hyperref}
\hypersetup{
  pdftitle={Optimierung f�r Studierende der Informatik, Aufgabenblatt 8},
  pdfauthor={Thomas Andreae},
  pdfcreator={LaTeX2e},
	hyperfootnotes=false,
	breaklinks=true,
	colorlinks=true,
	allcolors={black}}

\setlength{\parindent}{0em}
\pagestyle{empty}

\begin{document}

\begin{center}
\begin{Large}
\textbf{Optimierung f�r Studierende der Informatik}
\end{Large}

\textbf{Thomas Andreae}
	
\vspace{0.5cm}

\textbf{Wintersemester 2017/18}

\textbf{Blatt 8}

\vspace{0.5cm}
\end{center}

\small

\begin{enumerate}[\bfseries A:]

%------------------------------------------------------------------------
% Pr�senzaufgaben
%------------------------------------------------------------------------

\item \textbf{Pr�senzaufgaben am 11./12. Dezember 2017}

\begin{enumerate}[\bfseries 1.]

% Aufgabe P-1
\item Schauen Sie sich Figur 9.8 (Skript, Seite 127) an und denken Sie sich die Markierungen weg.
\begin{enumerate}[a)]
% Aufgabe P-1a
\item Erl�utern Sie Schritt f�r Schritt den Markierungsprozess: In welcher Reihenfolge wurden die Markierungen (labels) hinzugef�gt und was geben die Markierungen an?

\textbf{Hinweis}: Beachten Sie immer ($5^\prime$) (Skript, Seite 122)! F�r die Zeilen (6)--(9) bzw. (10)--(13) gilt die �bliche Regel: Gibt es mehrere Kandidaten f�r den n�chsten zu markierenden Knoten, so ist die alphabetische Reihenfolge entscheidend.

% Aufgabe P-1b
\item Wie kommt in Figur 9.8 der flussvergr��ernde Pfad $(s,a,b,c,f,t)$ zustande?
% Aufgabe P-1c
\item Wie kommt der in Figur 9.9 angegebene verbesserte Fluss zustande?
\end{enumerate}

% Aufgabe P-2
\item Wenden Sie den Algorithmus von Edmonds und Karp auf das folgende Netzwerk an. Gehen Sie dabei Schritt f�r Schritt vor, �hnlich wie im Beispiel auf den Seiten 123-128. (Es gelte wie �blich die Regel aus Aufgabe 1a).)

\begin{center}
\psset{xunit=1.00cm,yunit=1.00cm,linewidth=0.8pt,nodesep=0.5pt}
\begin{pspicture}(-0.5,-0.5)(6.5,4.5)

\cnode*(2,4){3pt}{A} \uput{0.25}[ 90](2,4){$a$}
\cnode*(2,2){3pt}{B} \uput{0.25}[270](2,2){$b$}
\cnode*(2,0){3pt}{C} \uput{0.25}[270](2,0){$c$}
\cnode*(4,0){3pt}{D} \uput{0.25}[270](4,0){$d$}
\cnode*(4,2){3pt}{E} \uput{0.25}[270](4,2){$e$}
\cnode*(0,2){3pt}{S} \uput{0.25}[180](0,2){$s$}
\cnode*(6,2){3pt}{T} \uput{0.25}[  0](6,2){$t$}

\ncline{->}{S}{A} \uput{0.05}[135](1.0, 3.0){$(3)$}
\ncline{->}{S}{B} \uput{0.10}[270](1.0, 2.0){$(5)$}
\ncline{->}{S}{C} \uput{0.05}[225](1.0, 1.0){$(6)$}
\ncline{->}{A}{B} \uput{0.05}[  0](2.0, 3.0){$(4)$}
\ncline{->}{A}{T} \uput{0.15}[ 90](4.0, 3.0){$(5)$}
\ncline{->}{B}{D} \uput{0.05}[215](3.0, 1.0){$(6)$}
\ncline{->}{B}{E} \uput{0.10}[270](3.0, 2.0){$(2)$}
\ncline{->}{C}{D} \uput{0.10}[270](3.0, 0.0){$(4)$}
\ncline{->}{D}{T} \uput{0.05}[315](5.0, 1.0){$(5)$}
\ncline{->}{E}{T} \uput{0.10}[270](5.0, 2.0){$(3)$}

\end{pspicture}
\end{center}

\end{enumerate}

%------------------------------------------------------------------------
% Hausaufgaben
%------------------------------------------------------------------------

\item \textbf{Hausaufgaben zum 18./19. Dezember 2017}

\begin{enumerate}[\bfseries 1.]

% Aufgabe H-1
\item Wie Pr�senzaufgabe 2 f�r das folgende Netzwerk:

\begin{center}
\psset{xunit=1.00cm,yunit=1.00cm,linewidth=0.8pt,nodesep=0.5pt}
\begin{pspicture}(-0.5,-0.5)(6.5,4.5)

\cnode*(2,4){3pt}{A} \uput{0.25}[ 90](2,4){$a$}
\cnode*(2,2){3pt}{B} \uput{0.25}[270](2,2){$b$}
\cnode*(2,0){3pt}{C} \uput{0.25}[270](2,0){$c$}
\cnode*(4,0){3pt}{D} \uput{0.25}[270](4,0){$d$}

\cnode*(4,3){3pt}{F} \uput{0.25}[ 90](4,3){$e$}
\cnode*(0,2){3pt}{S} \uput{0.25}[180](0,2){$s$}
\cnode*(6,2){3pt}{T} \uput{0.25}[  0](6,2){$t$}

\ncline{->}{S}{A} \uput{0.05}[135](1.0, 3.0){$(5)$}
\ncline{->}{S}{B} \uput{0.10}[270](1.0, 2.0){$(5)$}
\ncline{->}{S}{C} \uput{0.05}[225](1.0, 1.0){$(6)$}

\ncline{->}{A}{F} \uput{0.15}[ 90](3.0, 3.5){$(3)$}
\ncline{->}{B}{D} \uput{0.05}[215](3.0, 1.0){$(6)$}

\ncline{->}{B}{F} \uput{0.10}[ 90](3.0, 2.5){$(2)$}
\ncline{->}{C}{D} \uput{0.10}[270](3.0, 0.0){$(4)$}
\ncline{->}{D}{T} \uput{0.05}[315](5.0, 1.0){$(5)$}

\ncline{->}{F}{T} \uput{0.15}[ 90](5.0, 2.5){$(4)$}

\end{pspicture}
\end{center}

\textbf{Hinweis}: Beachten Sie immer ($5^\prime$) (Skript, Seite 122)! F�r die Zeilen (6)--(9) bzw. (10)--(13) gilt die �bliche Regel: Gibt es mehrere Kandidaten f�r den n�chsten zu markierenden Knoten, so ist die alphabetische Reihenfolge entscheidend.

\pagebreak

% Aufgabe H-2
\item Bearbeiten Sie die folgende Klausuraufgabe aus dem WS 2015/16:

\begin{enumerate}[a)]
% Aufgabe H-2a
\item Wir betrachten das folgende Netzwerk $N$, in dem die Kapazit�ten in Klammern angegeben sind; die Zahlen ohne Klammern bezeichnen den aktuellen Fluss, den wir $f_1$ nennen. Wie �blich seien $s$ und $t$ die Quelle bzw. die Senke des Netzwerks.

\begin{center}
\psset{xunit=1.25cm,yunit=1.25cm,linewidth=0.8pt,nodesep=0.5pt}
\begin{pspicture}(-0.5,-0.75)(8.5,4.75)
\small

\cnode*(0,2){3pt}{S} \uput{0.25}[180](0,2){$s$}
\cnode*(8,2){3pt}{T} \uput{0.25}[  0](8,2){$t$}
\cnode*(2,4){3pt}{A} \uput{0.25}[ 90](2,4){$a$}
\cnode*(2,2){3pt}{B} \uput{0.25}[270](2,2){$b$}
\cnode*(4,2){3pt}{C} \uput{0.25}[270](4,2){$d$}
\cnode*(6,4){3pt}{D} \uput{0.25}[ 90](6,4){$e$}
\cnode*(6,3){3pt}{E} \uput{0.25}[270](6,3){$f$}
\cnode*(2,0){3pt}{F} \uput{0.25}[270](2,0){$c$}

\footnotesize
\ncline{->}{S}{A} \uput{0.05}[135](1,3.0){$26(38)$}
\ncline{->}{S}{B} \uput{0.05}[270](1,2.0){$4(4)$}
\ncline{->}{S}{F} \uput{0.05}[225](1,1.0){$2(2)$}
\ncline{->}{A}{B} \uput{0.05}[  0](2,3.0){$8(8)$}
\ncline{->}{A}{C} \uput{0.25}[ 90](3,3.0){$9(10)$}
\ncline{->}{A}{D} \uput{0.05}[ 90](4,4.0){$9(12)$}
\ncline{->}{B}{C} \uput{0.05}[270](3,2.0){$14(26)$}
\ncline{->}{C}{E} \uput{0.05}[315](5,2.5){$8(8)$}
\ncline{->}{C}{F} \uput{0.05}[315](3,1.0){$11(24)$}
\ncline{->}{C}{T} \uput{0.05}[270](6,2.0){$4(4)$}
\ncline{->}{D}{B} \uput{0.10}[ 90](4,3.0){$2(3)$}
\ncline{->}{D}{E} \uput{0.05}[180](6,3.5){$0(4)$}
\ncline{->}{D}{T} \uput{0.05}[ 45](7,3.0){$7(7)$}
\ncline{->}{E}{T} \uput{0.05}[225](7,2.5){$8(9)$}
\ncline{->}{F}{T} \uput{0.05}[315](5,1.0){$13(27)$}

\normalsize
\end{pspicture}
\end{center}

Es soll der \textit{Algorithmus von Edmonds und Karp} angewendet werden, wobei nur die n�chste Flussvergr��erung betrachtet wird. Um $f_1$ zu verbessern, werden auf die �bliche Art Knotenmarkierungen vorgenommen, beispielsweise erh�lt $s$ die Markierung $(-, \infty)$ und $a$ erh�lt die Markierung $(s, +, 12)$.
\begin{enumerate}[(i)]
\item In welcher Reihenfolge werden die Knoten markiert? (\textbf{Regel}: Ist diese Reihenfolge durch den Algorithmus von Edmonds und Karp nicht festgelegt, so ist die alphabetische Reihenfolge entscheidend.) Geben Sie f�r jeden Knoten die zugeh�rige Markierung an! Gibt es Knoten, die unmarkiert bleiben?
\item Geben Sie den zunehmenden Pfad $P$ an, der zur Flussvergr��erung f�hrt und geben Sie auch den verbesserten Fluss $f_2$ an. (Es gen�gt, $f_2(e)$ f�r diejenigen Kanten anzugeben, f�r die $f_2(e) \neq f_1(e)$ gilt.)
\end{enumerate}

% Aufgabe H-2b
\item F�hrt man den Ford-Fulkerson-Algorithmus zur Bestimmung eines Maximalflusses in einem Netzwerk aus, so erh�lt man im letzten Schritt einen minimalen Schnitt $(S, T)$. Woran erkennt man, welche Knoten zu $S$ und welche zu $T$ geh�ren? (Geben Sie eine kurze und pr�zise Antwort.)

\end{enumerate}
\end{enumerate}


\end{enumerate}
\end{document}