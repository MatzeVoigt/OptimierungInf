\documentclass[11pt, a4paper]{article}
\usepackage{amsmath}
\usepackage{amsfonts}
\usepackage{amssymb}
\usepackage[latin1]{inputenc}
\usepackage[ngerman]{babel}
\usepackage[babel,german=quotes]{csquotes}
\usepackage{fullpage}
\usepackage{paralist}
\usepackage{mathtools}
\usepackage{eurosym}

% Meta Information festlegen
\usepackage{hyperref}
\hypersetup{
  pdftitle={Optimierung f�r Studierende der Informatik, Aufgabenblatt 02},
  pdfauthor={Thomas Andreae},
  pdfcreator={LaTeX2e},
	hyperfootnotes=false,
	breaklinks=true,
	colorlinks=true,
	allcolors={black}}

\setlength{\parindent}{0em}
\pagestyle{empty}

\begin{document}

\begin{center}
\begin{Large}
\textbf{Optimierung f�r Studierende der Informatik}
\end{Large}

\textbf{Thomas Andreae}
	
\vspace{0.5cm}

\textbf{Wintersemester 2017/18}

\textbf{Blatt 2}

\vspace{0.5cm}
\end{center}

\small

\begin{enumerate}[\bfseries A:]

%------------------------------------------------------------------------
% Pr�senzaufgaben
%------------------------------------------------------------------------

\item \textbf{Pr�senzaufgaben am 30. Oktober 2017}

\begin{enumerate}[\bfseries 1.]

% Aufgabe P-1
\item Schauen Sie sich das \textbf{Handout zum Simplexverfahren} an und beantworten Sie die folgenden Fragen:
\begin{enumerate}[(i)]
% Aufgabe P-1 (i)
\item Wie kommt das Starttableau zustande?
% Aufgabe P-1 (ii)
\item Weshalb wurde in der 1. Iteration $x_1$ als Eingangsvariable gew�hlt? Wie kommt die Wahl von $x_4$ als Ausgangsvariable zustande?
% Aufgabe P-1 (iii)
\item Als Ergebnis der 1. Iteration erh�lt man ein neues Tableau. Wie kommt die 1. Zeile in diesem Tableau zustande? Wie ergeben sich die �brigen Zeilen (einschlie�lich der $z$-Zeile)?
% Aufgabe P-1 (iv)
\item Woran erkennt man, dass das Tableau am Ende der 2. Iteration optimal ist? Wie ergibt sich am Schluss die optimale L�sung?
% Aufgabe P-1 (v)
\item K�nnen Sie anhand der $z$-Zeile im letzten Tableau begr�nden, weshalb die gefundene L�sung tats�chlich optimal ist?
\end{enumerate}

% Aufgabe P-2
\item Wir betrachten das folgende LP-Problem, das mit dem Simplexverfahren gel�st werden soll; dabei ist genau wie im \textbf{Handout} vorzugehen. Insbesondere ist am Ende jeder Iteration das neue Tableau noch einmal �bersichtlich hinzuschreiben (wie im Handout).
\begin{align*}
\begin{alignedat}{5}
& \text{maximiere } & 4x_1 &\ + &\ x_2 &\ - &\ 3x_3 & & \\
& \rlap{unter den Nebenbedingungen} & & & & & & & \\
&& -2x_1 &\ - &\ 2x_2 &\ + &\ 3x_3 &\ \leq &\ 2 \\
&&  2x_1 &\ + &\  x_2 &\   &\      &\ \leq &\ 5 \\
&&   x_1 &\ - &\  x_2 &\ - &\ 5x_3 &\ \leq &\ 4 \\
&&&&&& \llap{$x_1,x_2,x_3$} &\ \geq &\ 0
\end{alignedat}
\end{align*}

\end{enumerate}

%------------------------------------------------------------------------
% Hausaufgaben
%------------------------------------------------------------------------

\item \textbf{Hausaufgaben zum 6./7. November 2017}

\textbf{Hinweis}: Es ist in allen Aufgaben genau wie im \textbf{Handout} vorzugehen. Insbesondere ist am Ende jeder Iteration das neue Tableau noch einmal �bersichtlich hinzuschreiben (wie im Handout).

\begin{enumerate}[\bfseries 1.]

% Aufgabe H-1
\item L�sen Sie die folgenden LP-Probleme mit dem Simplexverfahren:
\begin{enumerate}[a)]

% Aufgabe H-1a
\item \begin{align*}
\begin{alignedat}{6}
& \text{maximiere } & -2x_1 &\ + &\ 3x_2 &\ + &\ \frac{1}{2}x_3 & & \\
& \rlap{unter den Nebenbedingungen} & & & & & & & \\
&& -2x_1 &\ + &\ 3x_2 &\ - &\  x_3 &\ \leq &\ 2 \\
&&   x_1 &    &       &\ + &\ 2x_3 &\ \leq &\ 5 \\
&&  -x_1 &\ + &\  x_2 &    &       &\ \leq &\ 2 \\
&& & & & & \llap{$x_1, x_2, x_3$} &\ \geq &\ 0 
\end{alignedat}
\end{align*}

% Aufgabe H-1b
\item \begin{align*}
\begin{alignedat}{6}
& \text{maximiere } & 3x_1 &\ + &\ x_2 &\ - &\ 11x_3 &\ - &\ 9x_4 & & \\
& \rlap{unter den Nebenbedingungen} & & & & & & & & & \\
&& x_1 &\ - &\ x_2 &\ - &\ 7x_3 &\ - &\ 3x_4 &\ \leq &\ 1\ \\
&& x_1 &\ + &\ x_2 &\ + &\ 3x_3 &\ + &\  x_4 &\ \leq &\ 3\ \\
&& & & & & & & \llap{$x_1,x_2,x_3,x_4$} &\ \geq &\ 0.
\end{alignedat}
\end{align*}
\end{enumerate}


% Aufgabe H-2
\item \begin{enumerate}[a)]
	% Aufgabe H-2a
	\item L�sen Sie die folgende Aufgabe mit dem Simplexverfahren:
	\begin{align*}
	\begin{alignedat}{5}
	& \text{maximiere } & -x_1 &\ + &\ 3x_2 &\ + &\ x_3 & & \\
	& \rlap{unter den Nebenbedingungen} & & & & & & & \\
	&& -x_1 &\ + &\ 2x_2 &\ + &\ 2x_3 &\ \leq &\ 8\ \\
	&&  x_1 &\ - &\ 2x_2 &\ + &\ 3x_3 &\ \leq &\ 8\ \\
	&&  x_1 &\ - &\ 3x_2 &\ + &\  x_3 &\ \leq &\ 8\ \\
	&&&&&& \llap{$x_1,x_2,x_3$} &\ \geq &\ 0.
	\end{alignedat}
	\end{align*}
	
	% Aufgabe P-1b
	\item Falls das Verfahren mit dem Ergebnis \textbf{unbeschr�nkt} terminiert, so gebe man zul�ssige L�sungen an, f�r die der Zielfunktionswert $z$ die folgenden Werte annimmt: $z=20$, $z=1000$ sowie $z=1000000$.
\end{enumerate}

\end{enumerate}
\end{enumerate}
\end{document}