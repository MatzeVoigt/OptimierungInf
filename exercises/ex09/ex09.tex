\documentclass[11pt, a4paper]{article}
\usepackage{amsmath}
\usepackage{amsfonts}
\usepackage{amssymb}
\usepackage[utf8]{inputenc}
\usepackage[ngerman]{babel}
\usepackage[babel,german=quotes]{csquotes}
\usepackage{fullpage}
\usepackage{paralist}
\usepackage{pst-all}

% Meta Information festlegen
\usepackage{hyperref}
\hypersetup{
  pdftitle={Optimierung für Studierende der Informatik, Aufgabenblatt 9},
  pdfauthor={},
  pdfcreator={LaTeX2e},
	hyperfootnotes=false,
	breaklinks=true,
	colorlinks=true,
	allcolors={black}}

\setlength{\parindent}{0em}
\pagestyle{empty}

\begin{document}

\begin{center}
\begin{Large}
\textbf{Optimierung für Studierende der Informatik}
\end{Large}

\textbf{}
	
\vspace{0.5cm}

\textbf{Wintersemester 2019/20}

\textbf{Blatt 9}

\vspace{0.5cm}
\end{center}

\small

\begin{enumerate}[\bfseries A:]

%------------------------------------------------------------------------
% Präsenzaufgaben
%------------------------------------------------------------------------

\item \textbf{Präsenzaufgaben am 18./19. Dezember 2017}

\begin{enumerate}[\bfseries 1.]

% Aufgabe P-1
\item Geben Sie einen zusammenhängenden Graphen $G$ mit 10 Kanten an, für den $m(G) = 3$ gilt.

% Aufgabe P-2
\item Für den Graphen auf Seite 143 des Skripts: Besitzt dieser Graph ein perfektes Matching? Falls ja, so gebe man ein solches an; falls nein, so begründe man, wieso kein perfektes Matching existiert.

% Aufgabe P-3
\item \begin{enumerate}[a)]
% Aufgabe P-3a
\item Wie ist die Knotenüberdeckungszahl $c(G)$ eines Graphen $G$ definiert?
% Aufgabe P-3b
\item Begründen Sie (kurz), weshalb $m(G) \leq c(G)$ für jeden Graphen $G$ gilt.
% Aufgabe P-3c
\item Geben Sie einen Graphen $G$ an, für den $m(G) < c(G)$ gilt.
% Aufgabe P-3d
\item Kann die Differenz $c(G) - m(G)$ beliebig groß werden?
\end{enumerate}


% Aufgabe P-4
\item Wahr oder falsch: Jeder Baum ist ein bipartiter Graph.

\end{enumerate}

%------------------------------------------------------------------------
% Hausaufgaben
%------------------------------------------------------------------------

\item \textbf{Hausaufgaben zum 8./9. Januar 2018}

\begin{enumerate}[\bfseries 1.]

% Aufgabe H-1
\item Bestimmen Sie für den unten angegebenen Graph ein Matching mit maximaler Kantenzahl sowie eine minimale Knotenüberdeckung. Verwenden Sie hierzu den Algorithmus von Edmonds und Karp, wobei die folgende Regel zu beachten ist: Gibt es mehrere Kandidaten für den nächsten zu markierenden Knoten, so sind Knoten mit kleinerem Index vorzuziehen.

\textbf{Hinweis}: Gehen Sie vor wie in Abschnitt 11.5.

\begin{center}
\psset{xunit=2.00cm,yunit=1.00cm,linewidth=0.8pt}
\begin{pspicture}(-0.5,-0.5)(6.5,2.5)

\cnode*(0,2){3pt}{X1} \uput{0.25}[ 90](0,2){$x_1$}
\cnode*(1,2){3pt}{X2} \uput{0.25}[ 90](1,2){$x_2$}
\cnode*(2,2){3pt}{X3} \uput{0.25}[ 90](2,2){$x_3$}
\cnode*(3,2){3pt}{X4} \uput{0.25}[ 90](3,2){$x_4$}
\cnode*(4,2){3pt}{X5} \uput{0.25}[ 90](4,2){$x_5$}
\cnode*(5,2){3pt}{X6} \uput{0.25}[ 90](5,2){$x_6$}
\cnode*(6,2){3pt}{X7} \uput{0.25}[ 90](6,2){$x_7$}
\cnode*(0,0){3pt}{Y1} \uput{0.25}[270](0,0){$y_1$}
\cnode*(1,0){3pt}{Y2} \uput{0.25}[270](1,0){$y_2$}
\cnode*(2,0){3pt}{Y3} \uput{0.25}[270](2,0){$y_3$}
\cnode*(3,0){3pt}{Y4} \uput{0.25}[270](3,0){$y_4$}
\cnode*(4,0){3pt}{Y5} \uput{0.25}[270](4,0){$y_5$}
\cnode*(5,0){3pt}{Y6} \uput{0.25}[270](5,0){$y_6$}

\ncline{-}{X1}{Y1}
\ncline{-}{X1}{Y2}
\ncline{-}{X1}{Y6}
\ncline{-}{X2}{Y1}
\ncline{-}{X2}{Y2}
\ncline{-}{X2}{Y3}
\ncline{-}{X2}{Y5}
\ncline{-}{X3}{Y1}
\ncline{-}{X3}{Y2}
\ncline{-}{X3}{Y4}
\ncline{-}{X4}{Y1}
\ncline{-}{X4}{Y3}
\ncline{-}{X4}{Y4}
\ncline{-}{X4}{Y5}
\ncline{-}{X5}{Y1}
\ncline{-}{X5}{Y4}
\ncline{-}{X5}{Y6}
\ncline{-}{X6}{Y1}
\ncline{-}{X6}{Y4}
\ncline{-}{X6}{Y6}
\ncline{-}{X7}{Y2}
\ncline{-}{X7}{Y4}

\end{pspicture}
\end{center}




% Aufgabe H-2
\item Wie Aufgabe 1 für den folgenden Graph:

\begin{center}
\psset{xunit=2.00cm,yunit=1.00cm,linewidth=0.8pt}
\begin{pspicture}(-0.5,-0.5)(6.5,2.5)

\cnode*(0,2){3pt}{X1} \uput{0.25}[ 90](0,2){$x_1$}
\cnode*(1,2){3pt}{X2} \uput{0.25}[ 90](1,2){$x_2$}
\cnode*(2,2){3pt}{X3} \uput{0.25}[ 90](2,2){$x_3$}
\cnode*(3,2){3pt}{X4} \uput{0.25}[ 90](3,2){$x_4$}
\cnode*(4,2){3pt}{X5} \uput{0.25}[ 90](4,2){$x_5$}
\cnode*(5,2){3pt}{X6} \uput{0.25}[ 90](5,2){$x_6$}
\cnode*(6,2){3pt}{X7} \uput{0.25}[ 90](6,2){$x_7$}
\cnode*(0,0){3pt}{Y1} \uput{0.25}[270](0,0){$y_1$}
\cnode*(1,0){3pt}{Y2} \uput{0.25}[270](1,0){$y_2$}
\cnode*(2,0){3pt}{Y3} \uput{0.25}[270](2,0){$y_3$}
\cnode*(3,0){3pt}{Y4} \uput{0.25}[270](3,0){$y_4$}
\cnode*(4,0){3pt}{Y5} \uput{0.25}[270](4,0){$y_5$}
\cnode*(5,0){3pt}{Y6} \uput{0.25}[270](5,0){$y_6$}

\ncline{-}{X1}{Y1}
\ncline{-}{X1}{Y2}
\ncline{-}{X1}{Y3}
\ncline{-}{X1}{Y4}
\ncline{-}{X1}{Y5}
\ncline{-}{X2}{Y1}
\ncline{-}{X2}{Y2}
\ncline{-}{X2}{Y3}
\ncline{-}{X2}{Y5}
\ncline{-}{X3}{Y1}
\ncline{-}{X3}{Y2}
\ncline{-}{X3}{Y4}
\ncline{-}{X4}{Y2}
\ncline{-}{X4}{Y4}
\ncline{-}{X5}{Y2}
\ncline{-}{X5}{Y6}
\ncline{-}{X6}{Y1}
\ncline{-}{X6}{Y4}
\ncline{-}{X6}{Y6}
\ncline{-}{X7}{Y1}
\ncline{-}{X7}{Y4}

\end{pspicture}
\end{center}


\end{enumerate}

\end{enumerate}
\end{document}
