\documentclass[11pt, a4paper]{article}
\usepackage{amsmath}
\usepackage{amsfonts}
\usepackage{amssymb}
\usepackage[utf8]{inputenc}
\usepackage[ngerman]{babel}
\usepackage[babel,german=quotes]{csquotes}
\usepackage{fullpage}
\usepackage{paralist}
\usepackage{pst-all}

% Meta Information festlegen
\usepackage{hyperref}
\hypersetup{
  pdftitle={Optimierung für Studierende der Informatik, Aufgabenblatt 10},
  pdfauthor={},
  pdfcreator={LaTeX2e},
	hyperfootnotes=false,
	breaklinks=true,
	colorlinks=true,
	allcolors={black}}

\setlength{\parindent}{0em}
\pagestyle{empty}

\begin{document}

\begin{center}
\begin{Large}
\textbf{Optimierung für Studierende der Informatik}
\end{Large}

\textbf{}
	
\vspace{0.5cm}

\textbf{Wintersemester 2019/20}

\textbf{Blatt 10}

\vspace{0.5cm}
\end{center}

\small

\begin{enumerate}[\bfseries A:]

%------------------------------------------------------------------------
% Präsenzaufgaben
%------------------------------------------------------------------------

\item \textbf{Präsenzaufgaben am 8./9. Januar 2018}

\begin{enumerate}[\bfseries 1.]

% Aufgabe P-1
\item Beim Entwurf von Approximationsalgorithmen spielen neben LP-basierten Methoden auch \textit{Greedy-Verfahren} eine wichtige Rolle. Im Folgenden wird ein Beispiel betrachtet, das dies illustriert.

\medskip

Ein Schiff mit $n$ Containern $1,2,\ldots,n$ erreicht den Hafen. Die Abmessungen der Container spielen keine Rolle, es geht ums Gewicht: Container $i$ habe das Gewicht $w_i > 0$ ($i=1,\ldots,n$). Zum Weitertransport stehen Lastwagen bereit, von denen jeder einen oder mehrere Container aufnehmen kann. \textit{Einzige Restriktion}: Es gibt eine Gewichtsschranke $K$, die für jeden Laster gilt, d.h., kein Lastwagen darf Container von einem Gesamtgewicht größer $K$ aufnehmen. Die Schranke $K$ und auch die Gewichte $w_i$ sollen als ganzzahlig angenommen werden. Es gelte $K \geq w_i$ für $i=1,\ldots,n$. Zu minimieren ist die Anzahl der Lastwagen, die zum Weitertransport aller Container benötigt werden. (Anmerkung: Das beschriebene Problem ist ein NP-schweres Optimierungsproblem.)

\textit{Ein Greedy-Algorithmus}: Die Container werden in der Reihenfolge $1,2,\ldots, n$ verladen, wobei immer nur ein Lastwagen zur Zeit beladen wird. Immer, wenn der nächste Container nicht mehr aufgeladen werden kann (wegen Überschreitung von $K$), wird ein Lastwagen für \enquote{voll} erklärt und auf die Reise geschickt.

\medskip

Mit $m^\star$ sei das optimale Ergebnis bezeichnet, d.h., $m^\star$ ist die minimale Anzahl der benötigten Lastwagen. Das Ergebnis, das der Greedy-Algorithmus liefert, werde mit $m$ bezeichnet. 

\begin{enumerate}[a)]
% Aufgabe P-1a
\item Belegen Sie anhand eines Beispiels, dass der Greedy-Algorithmus nicht immer das bestmögliche Ergebnis liefert. Mit anderen Worten: $m > m^\star$ ist möglich.

% Aufgabe P-1b
\item \textbf{Behauptung}: Es gilt immer $m < 2m^\star$. (Dies bedeutet, dass unser Greedy-Algorithmus gar nicht so schlecht ist: Es handelt sich um einen \textit{2-Approximationsalgorithmus}.)

\smallskip

Zeigen Sie die Richtigkeit dieser Behauptung für den Fall, dass $m$ ungerade ist.

\textbf{Hinweis}: $L_i$ sei das Gewicht, das der Greedy-Algorithmus auf den $i$-ten Lastwagen packt ($i=1,\ldots,m$). Welche naheliegende Feststellung lässt sich für die Summe $L_1+L_2$ und die Schranke $K$ treffen?
\end{enumerate}



% Aufgabe P-2
\item Im Skript wurde auf Seite 166 ein 2-Approximationsalgorithmus für das Knoten\-über\-deckungs\-problem vorgestellt.
\begin{enumerate}[a)]
% Aufgabe P-2a
\item Beschreiben Sie kurz in eigenen Worten, worum es beim Knotenüberdeckungsproblem geht.
% Aufgabe P-2b
\item Beschreiben Sie kurz (ebenfalls in eigenen Worten), wie der erwähnte 2-Approximations\-al\-go\-rith\-mus funktioniert.
% Aufgabe P-2c
\item Begründen Sie kurz, weshalb für die vom Algorithmus gelieferte Knotenüberdeckung $U$ gilt: $|U| \leq 2c(G)$.
% Aufgabe P-2d
\item Was versteht man unter dem vollständig bipartiten Graphen $K_{n,n}$?
\end{enumerate}

\end{enumerate}

%------------------------------------------------------------------------
% Hausaufgaben
%------------------------------------------------------------------------

\item \textbf{Hausaufgaben zum 15./16. Januar 2018}

\begin{enumerate}[\bfseries 1.]

% Aufgabe H-1
\item Wir betrachten das Lastwagenproblem aus Präsenzaufgabe 1 und verwenden die dort eingeführten Bezeichnungen.
\begin{enumerate}[a)]
% Aufgabe H-1a
\item Zeigen Sie die Richtigkeit der Behauptung aus Präsenzaufgabe 1b) für den Fall, dass $m$ gerade ist.

% Aufgabe H-1b
\item Es sei $k \geq 2$ eine ganze Zahl. Zeigen Sie, dass es für jedes derartige $k$ ein Beispiel gibt, für das $m^\star = k$ und $m=2k-1$ gilt. 

\textbf{Hinweis}: Lösen Sie zunächst die Fälle $k=2,3,4$ und orientieren Sie sich an diesen Fällen.

% Aufgabe H-1c
\item Begründen Sie kurz, weshalb aus b) folgt, dass unser Algorithmus \textit{kein} $\gamma$-Approximations\-al\-go\-rith\-mus für $\gamma < 2$ ist.
\end{enumerate}


% Aufgabe H-2
\item \begin{enumerate}[a)]
% Aufgabe H-2a
\item Auf Seite 166 des Skripts wurde im Zusammenhang mit dem Knotenüberdeckungsproblem die folgende Frage gestellt: \textit{Ist es möglich, den gefundenen Faktor 2 zu verbessern, indem man den Algorithmus unverändert lässt, aber eine raffiniertere Analyse des Algorithmus vornimmt?} 

\smallskip
Zeigen Sie mithilfe des vollständig bipartiten Graphen $K_{n,n}$, dass dies nicht möglich ist.

% Aufgabe H-2b
\item Es sei $k \geq 2$ eine fest gewählte ganze Zahl. Gegeben sei eine Menge $S$ und eine Kollektion $T_1,\ldots,T_m$ von $k$-elementigen Teilmengen von $S$. Es gelte also
\[
T_i \subseteq S \text{  und  } |T_i|=k \quad (i=1,\ldots,m).
\]

Eine Teilmenge $H \subseteq S$ wird ein \textit{Hitting Set} genannt, falls alle Mengen $T_i$ von $H$ getroffen werden, d.h., falls $H \cap T_i \neq \emptyset$ für alle $i = 1,\ldots,m$ gilt. Gesucht ist ein Hitting Set mit einer minimalen Anzahl von Elementen. (Mitteilung: Dies ist ein NP-schweres Optimierungsproblem.) Wir nennen das beschriebene Problem $k$-HITTING SET. Das Problem lässt sich also wie folgt beschreiben:


\medskip
\begin{center}
\begin{minipage}{0.85\textwidth}
\textbf{$\mathbf{k}$-HITTING SET}

\medskip
\textbf{Eingabe}: eine Menge $S$ sowie eine Kollektion von $k$-elementigen Teilmengen von $S$.

\medskip
\textbf{Gesucht}: ein Hitting Set mit einer minimalen Anzahl von Elementen.
\end{minipage}
\end{center}

\medskip
Man beachte: $k$ ist \textit{nicht} Teil der Eingabe, sondern eine \textit{Konstante}.


\begin{enumerate}[(i)]
% Aufgabe H-2b (i)
\item Beschreiben Sie einen $k$-Approximationsalgorithmus für $k$-HITTING SET.

% Aufgabe H-2b (ii)
\item Weisen Sie nach, dass der von Ihnen unter (i) vorgeschlagene Algorithmus tatsächlich ein $k$-Approximationsalgorithmus ist.
\end{enumerate}

\textbf{Hinweis zu (i) und (ii)}: Denken Sie zunächst an den Spezialfall $k=2$. Sie kennen bereits einen 2-Approximationsalgorithmus für diesen Spezialfall: siehe Abschnitt 11.7.1 im Skript.

\end{enumerate}




\end{enumerate}

\end{enumerate}
\end{document}