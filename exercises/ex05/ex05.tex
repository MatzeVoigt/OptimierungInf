\documentclass[11pt, a4paper]{article}
\usepackage{amsmath}
\usepackage{amsfonts}
\usepackage{amssymb}
\usepackage[utf8]{inputenc}
\usepackage[ngerman]{babel}
\usepackage[babel,german=quotes]{csquotes}
\usepackage{fullpage}
\usepackage{paralist}
\usepackage{mathtools}

% Meta Information festlegen
\usepackage{hyperref}
\hypersetup{
  pdftitle={Optimierung für Studierende der Informatik, Aufgabenblatt 05},
  pdfauthor={},
  pdfcreator={LaTeX2e},
	hyperfootnotes=false,
	breaklinks=true,
	colorlinks=true,
	allcolors={black}}

\setlength{\parindent}{0em}
\pagestyle{empty}

\begin{document}

\begin{center}
\begin{Large}
\textbf{Optimierung für Studierende der Informatik}
\end{Large}

\textbf{}
	
\vspace{0.5cm}

\textbf{Wintersemester 2019/20}

\textbf{Blatt 5}

\vspace{0.5cm}
\end{center}

\small

\begin{enumerate}[\bfseries A:]

%------------------------------------------------------------------------
% Präsenzaufgaben
%------------------------------------------------------------------------

\item \textbf{Präsenzaufgaben am 20./21. November 2017}

\begin{enumerate}[\bfseries 1.]

% Aufgabe P-1
\item Wir greifen das 2. Beispiel (\enquote{Second Example}) aus Kapitel 2 auf (Skript, Seite 23) und nennen es $(P)$.
\begin{enumerate}[(i)]
% Aufgabe P-1 (i)
\item Stellen Sie das zugehörige duale Problem $(D)$ auf.
% Aufgabe P-1 (ii)
\item Eine optimale Lösung $(x_1^*, x_2^*, x_3^*)$ für $(P)$ haben wir bereits mit dem Simplexverfahren bestimmt. Lesen Sie zusätzlich eine optimale Lösung $(y_1^*, y_2^*, y_3^*, y_4^*)$ für $(D)$ am letzten Tableau ab.
% Aufgabe P-1 (iii)
\item Überprüfen Sie, ob die von Ihnen abgelesene Lösung $(y_1^*, y_2^*, y_3^*, y_4^*)$ tatsächlich eine \textit{zulässige} Lösung von $(D)$ ist.
% Aufgabe P-1 (iv)
\item Überprüfen Sie mithilfe des Dualitätssatzes, ob $(y_1^*, y_2^*, y_3^*, y_4^*)$ tatsächlich eine \textit{optimale} Lösung von $(D)$ ist.
% Aufgabe P-1 (v)
\item Bestätigen Sie noch einmal, dass es sich bei $(x_1^*, x_2^*, x_3^*)$ und $(y_1^*, y_2^*, y_3^*, y_4^*)$ um optimale Lösungen von (P) bzw. (D) handelt, indem Sie zeigen, dass die komplementären Schlupfbedingungen (Satz 3, Skript Seite 76) erfüllt sind. 

\end{enumerate}

%% Aufgabe P-2
%\item Wir betrachten das folgende LP-Problem, das wir $(P)$ nennen:
%\begin{align*}
%\begin{alignedat}{4}
%& \text{maximiere } & -x_1 &\ - &\ 2x_2 & & \\
%& \rlap{unter den Nebenbedingungen} & & & & & \\
%&& -3x_1 &\ + &\ x_2 &\ \leq &\ -1\ \\
%&&   x_1 &\ - &\ x_2 &\ \leq &\ 1\ \\
%&& -2x_1 &\ + &\ 7x_2 &\ \leq &\ 6\ \\
%&&  9x_1 &\ - &\ 4x_2 &\ \leq &\ 6\ \\
%&& -5x_1 &\ + &\ 2x_2 &\ \leq &\ -3\ \\
%&&  7x_1 &\ - &\ 3x_2 &\ \leq &\ 6\ \\
%&& && \llap{$x_1,x_2$} &\ \geq &\ 0.
%\end{alignedat}
%\end{align*}
%
%Lösen Sie $(P)$, indem Sie das zu $(P)$ duale Problem $(D)$ aufstellen und $(D)$ mit dem Simplexverfahren lösen.

\end{enumerate}

%------------------------------------------------------------------------
% Hausaufgaben
%------------------------------------------------------------------------

\item \textbf{Hausaufgaben zum 27./28. November 2017}

\begin{enumerate}[\bfseries 1.]

% Aufgabe H-1
\item \begin{enumerate}[a)]

% Aufgabe H-1a
\item Wir greifen das Beispiel aus Hausaufgabe 1a) von Blatt 2 auf und nennen es $(P)$.
\begin{enumerate}[(i)]
% Aufgabe H-1a (i)
\item Stellen Sie das zugehörige duale Problem $(D)$ auf.
% Aufgabe H-1a (ii)
\item Eine optimale Lösung $(x_1^*, x_2^*, x_3^*)$ für $(P)$ haben wir bereits mit dem Simplexverfahren bestimmt. Lesen Sie zusätzlich eine optimale Lösung $(y_1^*, y_2^*, y_3^*)$ für $(D)$ am letzten Tableau ab.
% Aufgabe H-1a (iii)
\item Überprüfen Sie, ob die von Ihnen abgelesene Lösung $(y_1^*, y_2^*, y_3^*)$ tatsächlich eine \textit{zulässige} Lösung von $(D)$ ist.
% Aufgabe H-1a (iv)
\item Überprüfen Sie mithilfe des Dualitätssatzes, ob $(y_1^*, y_2^*, y_3^*)$ tatsächlich eine \textit{optimale} Lösung von $(D)$ ist.
% Aufgabe H-1a (v)
\item Bestätigen Sie noch einmal, dass es sich bei $(x_1^*, x_2^*, x_3^*)$ und $(y_1^*, y_2^*, y_3^*)$ um optimale Lösungen von (P) bzw. (D) handelt, indem Sie zeigen, dass die komplementären Schlupfbedingungen (Satz 3, Skript Seite 76) erfüllt sind. 
\end{enumerate}

% Aufgabe H-1b
\item Wie a) für Hausaufgabe 1b) von Blatt 2.

\end{enumerate}


% Aufgabe H-2
\item \begin{enumerate}[a)]
% Aufgabe H-2a
\item Schauen Sie sich die in Abschnitt 7.4 im Anschluss an Satz 3' aufgeführten Beispiele 1 und 2 an (Skript Seite 78 f.) und bearbeiten Sie die auf Seite 79 gestellte Aufgabe.

% Aufgabe H-2b
\item Gegeben sei das folgende LP-Problem (P) zusammen mit einer vorgeschlagenen Lösung:
\begin{align*}
\begin{alignedat}{5}
& \text{maximiere } & 3x_1 &\ + &\ 2x_2 &\ + &\ 4x_3 & & \\
& \rlap{unter den Nebenbedingungen} & & & & & & & \\
&&   x_1 &\ + &\ x_2 &\ + &\ 2x_3 &\ \leq &\ 4\ \\
&&  2x_1 &\   &\     &\ + &\ 3x_3 &\ \leq &\ 5\ \\
&&  2x_1 &\ + &\ x_2 &\ + &\ 3x_3 &\ \leq &\ 7\ \\
&& &&&& \llap{$x_1,x_2,x_3$} &\ \geq &\ 0.
\end{alignedat}
\end{align*}

Vorgeschlagene Lösung:
\[
x_1^*= \frac{5}{2}, \quad
x_2^*= \frac{3}{2}, \quad
x_3^*= 0.
\]

Prüfen Sie mithilfe von Satz 3' (Skript Seite 78), ob dies eine optimale Lösung von (P) ist.
\end{enumerate}

\end{enumerate}
\end{enumerate}
\end{document}